%%%% Proceedings format for most of ACM conferences (with the exceptions listed below) and all ICPS volumes.
\documentclass[sigconf]{acmart}
%%%% As of March 2017, [siggraph] is no longer used. Please use sigconf (above) for SIGGRAPH conferences.

%%%% Proceedings format for SIGPLAN conferences 
% \documentclass[sigplan, anonymous, review]{acmart}

%%%% Proceedings format for SIGCHI conferences
% \documentclass[sigchi, review]{acmart}

%%%% To use the SIGCHI extended abstract template, please visit
% https://www.overleaf.com/read/zzzfqvkmrfzn

%
% defining the \BibTeX command - from Oren Patashnik's original BibTeX documentation.
\def\BibTeX{{\rm B\kern-.05em{\sc i\kern-.025em b}\kern-.08emT\kern-.1667em\lower.7ex\hbox{E}\kern-.125emX}}
    
% Rights management information. 
% This information is sent to you when you complete the rights form.
% These commands have SAMPLE values in them; it is your responsibility as an author to replace
% the commands and values with those provided to you when you complete the rights form.
%
% These commands are for a PROCEEDINGS abstract or paper.

\newcommand{\COMMENTBOX}[3]{\begin{center}\colorbox{#1}{\parbox{7cm}{#2: #3}}\end{center}}
\newcommand{\CC}[1]{\COMMENTBOX{yellow}{CC}{#1}}
\newcommand{\AR}[1]{\COMMENTBOX{blue!30}{AR}{#1}}
\newcommand{\JP}[1]{\COMMENTBOX{green!50}{JP}{#1}}
\renewcommand{\SS}[1]{\COMMENTBOX{red!30}{SS}{#1}}


\copyrightyear{2018}
\acmYear{2018}
\setcopyright{acmlicensed}
\acmConference[Woodstock '18]{Woodstock '18: ACM Symposium on Neural Gaze Detection}{June 03--05, 2018}{Woodstock, NY}
\acmBooktitle{Woodstock '18: ACM Symposium on Neural Gaze Detection, June 03--05, 2018, Woodstock, NY}
\acmPrice{15.00}
\acmDOI{10.1145/1122445.1122456}
\acmISBN{978-1-4503-9999-9/18/06}

%
% These commands are for a JOURNAL article.
%\setcopyright{acmcopyright}
%\acmJournal{TOG}
%\acmYear{2018}\acmVolume{37}\acmNumber{4}\acmArticle{111}\acmMonth{8}
%\acmDOI{10.1145/1122445.1122456}

%
% Submission ID. 
% Use this when submitting an article to a sponsored event. You'll receive a unique submission ID from the organizers
% of the event, and this ID should be used as the parameter to this command.
%\acmSubmissionID{123-A56-BU3}

%
% The majority of ACM publications use numbered citations and references. If you are preparing content for an event
% sponsored by ACM SIGGRAPH, you must use the "author year" style of citations and references. Uncommenting
% the next command will enable that style.
%\citestyle{acmauthoryear}

%
% end of the preamble, start of the body of the document source.
\begin{document}

%
% The "title" command has an optional parameter, allowing the author to define a "short title" to be used in page headers.
\title{Power Profiling and Analysis of Code Obfuscation for Embedded Systems}

%
% The "author" command and its associated commands are used to define the authors and their affiliations.
% Of note is the shared affiliation of the first two authors, and the "authornote" and "authornotemark" commands
% used to denote shared contribution to the research.


%
% By default, the full list of authors will be used in the page headers. Often, this list is too long, and will overlap
% other information printed in the page headers. This command allows the author to define a more concise list
% of authors' names for this purpose.
%\renewcommand{\shortauthors}{Trovato and Tobin, et al.}


\begin{abstract}
Code obfuscation provides time-limited protection of the confidentiality of data and software by garbling code, thus making reverse engineering harder. While the literature is rife with obfuscating code transformations, their application within the context of embedded systems poses considerable challenges, due to critical resource constraints, especially energy. In this paper we present a comprehensive study of the energy and performance impact of a wide range of obfuscation techniques for embedded devices. In particular, we leverage Tigress, a freely available source-to-source C language obfuscator and characterize the power consumed by transformations applied to the Mibench benchmark suite and comparatively analyze energy-performance trade-offs. Our initial studies reveal that, energy consumption and execution time increase by a maximum of twenty-fold and eight-fold respectively for the virtualization transformation compared to unobfuscated code. These insights enable software developers and software vendors to select appropriate obfuscating transformations depending on the energy, storage, and performance requirements of embedded applications. 
\end{abstract}
\keywords{obfuscation, tigress, energy, text tagging}

%
% A "teaser" image appears between the author and affiliation information and the body 
% of the document, and typically spans the page. 
%\begin{teaserfigure}
%  \includegraphics[width=\textwidth]{sampleteaser}
%  \caption{Seattle Mariners at Spring Training, 2010.}
%  \Description{Enjoying the baseball game from the third-base seats. Ichiro Suzuki preparing to bat.}
%  \label{fig:teaser}
%\end{teaserfigure}

\maketitle

\section{Introduction}
% MATE attacks against IoT devices
The Internet of Things (IoT) has seen an unprecedented growth, representing a new paradigm of devices interacting with each other, their environment, and the larger Internet~\cite{ATZORI20102787}. As these embedded devices have become ubiquitous, security and privacy have emerged as critical concerns~\cite{weber2010internet,7054433}. Since IoT devices run proprietary software and firmware and handle sensitive data, they become susceptible to attacks through tampering and reverse engineering. These types of attacks, where an adversary is in physical control of an IoT device and can manipulate its code, data, or hardware at will, are termed {\em Man-At-The-End} (MATE) attacks. 

% Software protection techniques.
Techniques to mitigate MATE attacks are termed {\em software protection} or {\em anti-tamper protection}~\cite{falcarin2011guest}. {\em Code Obfuscation} and {\em Software Tamperproofing} are popular techniques to protect code and data against MATE attacks~\cite{collberg_surreptitious_2010}. Experimental assessment of obfuscation has shown that it provides (time-limited~\cite{hohl98time}) protection against MATE attacks~\cite{5090041,7781792}.

% Performance issues for IoT devices.
It has been argued~\cite{Hosseinzadeh2015} that obfuscation can enhance code and data protection in the context of resource constrained devices as well. However, given the limited computational power available on IoT devices, any technique that purports to achieve secure computation on such devices must be evaluated with respect to its overhead: power, performance, and code size. This is a serious concern for obfuscation-based protection, as many obfuscating transformations can result in considerable computational overhead. Some prior work has addressed these issues in the context of simplistic code obfuscation techniques~\cite{6976079,dhukovic2015load,raj2017modelling}.
\subsection{}
% What we do in this work.
In this work, we study the relative impact of different obfuscating transformations on power and performance in a resource constrained environment. The ultimate goal is to make available data that will allow practitioners to select the right obfuscation technique given (a) the constraints of their hardware; (b) the characteristics of their software system; and (c) the type of asset they want to protect. Specifically, in this paper we present measurements of the energy consumption of the Raspberry Pi (RPi) running programs from the MiBench benchmark suite that have been obfuscated with the Tigress code obfuscation tool. 

Tigress~\cite{Collberg2012Distributed,banescu2015framework,banescu2016code} is a diversifying C-to-C source code obfuscator that supports a wide range of traditional obfuscating transformations. In our experimental setup we utilize the RPi as our IoT device and Tigress as our obfuscation tool. We use the MiBench embedded benchmark suite~\cite{990739} which is written in C and is designed to be representative of embedded software. We build a custom power measurement and data acquisition framework to measure the current drawn and power consumed by the various obfuscation techniques.

% Overview of rest of the paper
The rest of the paper is organized as follows. In Section~\ref{sec:related} we discuss related work in code obfuscation and IoT security. Section~\ref{sec:methodology} we describe our experimental methodology, including the obfuscating transformations we employ, and details of the experimental framework setup for power measurement and acquisition. We discuss our results in Section~\ref{sec:analysis} and outline directions for future work in Section~\ref{sec:discussion}.

% Sharing statement
{\em Sharing Statement:} The binary for the \tigress obfuscation tool can be downloaded from \tigressurl (source code is available to researchers on request). All source code, scripts, and benchmarks used to derive the results in this paper, along with raw and processed data sets are freely available to the research community at \sourceurl. 

%%%%%%%%%%%%%%%%%%%%%%%%%%%%%%%%%%%%%%%
% Old stuff
%%%%%%%%%%%%%%%%%%%%%%%%%%%%%%%%%%%%%%%
\endinput
Several works exist in literature that have analyzed and addressed the impact of security mechanisms (cryptographic algorithms and security protocols) in such devices. \CC{CITATION}   

Both open-source and commercial tools have been used to study the overall affect of obfuscation on energy, efficiency and quality of code. A taxonomy of well established obfuscation techniques was postulated~\cite{collberg1997taxonomy}. Little is studied of the impacts of these obfuscation techniques on resources relative to each other.



\section{Background and Related Work}\label{sec:rw}
\CC{Sriram writes a brief intro here}

\subsection{Code Obfuscation}
\CC{Christian writes this}

\subsection{Security Issues of IoT Devices}
\CC{Sriram writes this}

\subsection{Performance of IoT Devices}
Ensuring security and privacy in resource constraint devices (or IoTs) constraints is challenging~\cite{7823334, 6970594, 5675772}. Key constraints to be considered in such devices are power, execution time and size. A significant amount of work has been done in this direction regarding the impact from security protocols and cryptographic algorithms, mainly~\cite{potlapally2003analyzing, 5983970, 1347774, 5940923}. These works analyze the impact of energy, power and performance in the context of mobile device where battery life is the main constraint. The same concerns prevail while utilizing obfuscation techniques in such constraint devices, particularly IoTs, that handle sensitive data and run proprietary code requiring the need for code/data obfuscation to harden reverse engineering.
\CC{Aakarsh needs to update this. You need to say things like "In [], Bob investigates ..., and finds that ...}

\subsection{Obfuscation Performance Evaluation}
However, to our knowledge there is no work that analyzes the impact of a taxonomy of obfuscation techniques on power and performance in constraint devices. The closest related works are~\cite{6976079, Ceccato, dhukovic2015load}, where the impact of code obfuscation are studied based on code metrics, load-profiles and energy. The energy usage of different code obfuscation tools and configurations are empirically studied for mobile application~\cite{6976079}. However, i) the obfuscation configurations utilized are not well established and their work is of the impact of commercial obfuscation tools rather than a taxonomy of obfuscation techniques, and ii) their platform is a smartphone which have considerably lesser constraints compared to a typical IoT device.

\CC{Aakarsh, same thing here. You need to briefly explain what the authors did, and what they found.}


\section{Experimental Set-up}
\CC{Sriram, write a short introduction here}

\subsection{Obfuscating Transformations}



% Please add the following required packages to your document preamble:
% \usepackage{multirow}
\begin{table*}[!hp]
\caption{Tigress transformations and transformation options used in the experiments.}
\label{tab:tigress}
\begin{tabular}{|p{2.25cm}|p{4cm}|p{11cm}|}
\hline
\multicolumn{1}{|c|}{Transformation} & \multicolumn{1}{c|}{Description}                                                                                                                                                                                                                                                                        & \multicolumn{1}{c|}{Options}                                                                                                                                                                                                                                                                                                                             \\ \hline
\multirow{3}{*}{AddOpaque}           & \multirow{3}{4cm}{Split statement sequences by adding if-statement, protected by opaque predicates~\cite{collberg98manufacturing}.} & \underline{Count} \{1,5,10,15 and 20\}: Number of opaques to add to each function                                                                                                                                                                                                                                                                                             \\ \cline{3-3} 
                                     &                                                                                                                                                                                                                                                                                                         & \underline{Kinds} \{true,bug,call,junk\}: Insertion of bogus computation allowed. Where true is the real statement, bug is a buggy statement, call is a random function and junk is random bytes.                                                                                                                                                                     \\ \cline{3-3} 
                                     &                                                                                                                                                                                                                                                                                                         & \underline{Structs} \{list, array\}: Generate opaque expressions using linked lists or arrays.                                                                                                                                                                                                                                                                       \\ \hline
EncodeArithmetic                     & Replace integer arithmetic expression with more complex expression.                                                                                                                                                                                                                  &                                                                                                                                                                                                                                                                                            \\ \hline
EncodeLiteral                        & Obfuscate string and integer literals with less obvious expressions.                                                                                                                                                                                                                    & \underline{Kinds} \{integer, string\}: Integer replaces literal integers with opaque expressions and string replaces literal strings with calls to a function that generates them.                                                                                                                                                                                                                                                                                            \\ \hline
\multirow{6}{*}{Flatten}             & \multirow{6}{4cm}{Remove structured control flow using the control flow flattening~\cite{wang00security}.}                                                                                                                                                                                         & \underline{Dispatch} \{call, goto, switch, and indirect\}: Specifies the dispatch method of the flattened blocks of code. For call, each block is turned into its own function, goto uses indirect goto statements, and switch uses switch statements.                                                                                                       \\ \cline{3-3} 
                                     &                                                                                                                                                                                                                                                                                                         & \underline{SplitBasicBlocks} \{true,false\}: Flatten into smaller pieces by splitting up basic blocks prior to flattening.                                                                                                                                                              \\ \cline{3-3} 
                                     &                                                                                                                                                                                                                                                                                                         & \underline{RandomizeBlocks} \{true,false\}: If true, randomize  basic block sequences.                                                                                                                                                                                                                                                                 \\ \cline{3-3} 
                                     &                                                                                                                                                                                                                                                                                                         & \underline{ObfuscateNext} \{true,false\}: If true, obfuscate the computation of the dispatch variable using opaque expressions.                                                                                                                                                                                                                                        \\ \cline{3-3} 
                                     &                                                                                                                                                                                                                                                                                                         & \underline{OpaqueStructs} \{list,array\}: Type of opaque predicate to use which.                                                                                                                                                                                                                                                       \\ \cline{3-3} 
                                     &                                                                                                                                                                                                                                                                                                         & \underline{ConditionalKinds} \{branch,compute\}: How to transform conditional branches: either normal branches computed jumps.                                                                                                                                                                                                          \\ \hline
\multirow{7}{*}{Merge}               & \multirow{7}{4cm}{Merge multiple functions together into one, adding an extra formal argument to be able to call any of the constituent functions. Useful as a precursor to virtualization.
}                                                                                                                                                                                                             & Flatten \{true\}: Flatten before merging.                                                                                                                                                                                                                                                                                                                \\ \cline{3-3} 
                                     &                                                                                                                                                                                                                                                                                                         & \underline{ObfuscateSelect} \{true,false\}: If true, obfuscate the function selection argument with opaque expressions.                                                                                                                                                                                                               \\ \cline{3-3} 
                                     &                                                                                                                                                                                                                                                                                                         & \underline{OpaqueStructs} \{list,array\}: Type of opaque predicate to use.                                                                                                                                                                                                                                                       \\ \cline{3-3} 
                                     &                                                                                                                                                                                                                                                                                                         & \underline{Dispatch} \{indirect ,goto,switch\}: The dispatch method of the flattened merge.                                                                                                                       \\ \cline{3-3} 
                                     &                                                                                                                                                                                                                                                                                                         & \underline{RandomizeBlocks} \{true,false\}: If true, flattened basic block sequences are randomized.                                                                                                                                                                                                                                                                 \\ \cline{3-3} 
                                     &                                                                                                                                                                                                                                                                                                         & \underline{Split basic blocks} \{true,false\}: If true, basic blocks will be split up into individual blocks prior to merging.                                                                                                                                                                                                                                 \\ \cline{3-3} 
                                     &                                                                                                                                                                                                                                                                                                         & \underline{ConditionalKinds} \{branch,compute\}: How to transform conditional branches: either normal branches computed jumps.                                                                                                                                                                                                          \\ \hline
\multirow{2}{*}{Split}               & \multirow{2}{4cm}{Break a large function into smaller, less conspicuous, pieces which can be further obfuscated. }                                                                                                                                                                                                           & \underline{Count} \{1,10,15 and 20\}: Number of attempts to split the function.                                                                                                                                                                                                                                                                   \\ \cline{3-3} 
                                     &                                                                                                                                                                                                                                                                                                         & \underline{SplitKinds} \{block,inside,deep,top and recursive\}: the order in which the function is traversed when looking for statement sequences that can be outlined into a separate function. \\ \hline

\multirow{5}{*}{Virtualize}          & \multirow{5}{4cm}{Turn a function into an interpreter. The bytecode language is unique to this function. 
}                                                                                                                                                                                                                                                  & \underline{Dispatch} \{switch and indirect\}: The interpreter's dispatch method. {\em Indirect} uses {\tt gcc}'s {\em labels-as-values} to implement {\em indirect threaded dispatch}.                                                                                                                                                                                                                                                                               \\ \cline{3-3} 
                                     &                                                                                                                                                                                                                                                                                                         & \underline{SuperOpsRatio} \{0,2\}: Number of super operators~\cite{proebsting96optimizing}. If {\tt >0} randomly merges  instructions together to make highly diverse virtual instruction sets.                                                                                                                                                                                                                                                                                                      \\ \cline{3-3} 
                                     &                                                                                                                                                                                                                                                                                                         & \underline{MaxMergeLength} \{0,5\}: Longest sequence of instructions to be merged into one with superoperators.                                                                                                                                                                                                                                                                        \\ \cline{3-3} 
                                     &                                                                                                                                                                                                                                                                                                         & \underline{Performance} \{PointerStack,AddressSizeShort,CacheTop\}: Tweak performance:  AddressSizeShort uses pointer arithmetic to store instruction handler addresses in a 16-bit variable, CacheTop stores the top of stack in a register.                                                                  \\ \cline{3-3} 
                                     &                                                                                                                                                                                                                                                                                                         & \underline{Operands} \{stack, registers\}: Types of virtual instruction set operands: either (implicit) stack operands or (explicit) register operands, or both. Increases virtual instruction set diversity.                                                                                                                                                                                               \\ \hline
\end{tabular}
\end{table*}

In our experiments we use the Tigress C source-to-source obfuscator. Tigress supports a wide range of transformations, and each transformation has multiple options that controls how the obfuscated code is generated. Table~\ref{tab:tigress} shows the Tigress transformations, options, and range of option values that we used in our experiments.

\begin{figure*}
\begin{center}
\begin{minipage}{12cm}
\begin{lstlisting}[basicstyle=\footnotesize]
tigress --Seed=42 --Transform=InitOpaque --Functions=main \
   --Transform=Virtualize --Functions=foo --VirtualizeDispatch=direct \
   --Transform=AddOpaque --Functions=foo --AddOpaqueCount=20 --out=output.c input.c
\end{lstlisting}
\end{minipage}
\end{center}
\caption{Example Tigress invocation.}
\label{fig:tigresscall}
\end{figure*}

Figure~\ref{fig:tigresscall} shows a  call to Tigress to transform a C program {\tt input.c} into {\tt output.c}. This particular invocation first applies the {\tt Virtualize} transformation to the function named {\tt foo} using {\em direct} dispatch, and then adds 20 opaque predicates to the virtualized {\tt foo}. This is a very typical use of an obfuscator, applying multiple transformations to the same function, in order to protect against different types of attack. Note that a {\em seed} can be given to Tigress, which randomizes all internal decisions and allows multiple, differently obfuscated, programs to be generated from the same input program. This is used to generate a diverse set of programs, making it more difficult for an adversary to create generalized attack scripts that work across many obfuscated programs. 

In our experiments, we do not combine transformations, but only apply one at a time to each of the benchmarks. For each transformation we run all possible combinations of the values of all options shown in Table~\ref{tab:tigress}. This results in 16 variants for \emph{Virtualize}, 128 variants for \emph{Merge}, 35 variants for \emph{Split}, 75 variants for \emph{AddOpaque}, 128 variants for \emph{Flatten}, 1 variant for \emph{EncodeArithmetic} and 3 variants for \emph{EncodeLiteral}. 

We apply each transformation variant to all the functions in each benchmark. In practice, this is unrealistic; in most scenarios only security-sensitive functions are obfuscated to avoid too much overhead. For the MiBench benchmarks, however, it is unclear which part of the code should be considered ``security-sensitive,'' and we therefore transform the entire program.


After transformation, the resulting program is compiled with gcc version 4.6 using the command: \texttt{gcc -O2 -lm -Wall -o obfuscated.exe obfuscated\_input.c}. The resulting program is stripped of symbols.

When functions are not inlined, there is a performance hit to make a function call.
Function calls (depending on the platform) typically involve a few 10s of instructions, and that's including saving / restoring the stack. Some function calls consist a jump and return instruction.
But there are other things that might impact function call performance and energy as a result. The function being called may not be loaded into the processor's cache, causing a cache miss and forcing the memory controller to grab the function from main RAM. Hence function calls may or may not impact performance and energy.

%%%%%%%%%%%%%%%%%%%%%%%%%%%%%%%%%%%%%%
% Old stuff
%%%%%%%%%%%%%%%%%%%%%%%%%%%%%%%%%%%%%% 
\endinput

\AR{Dr.Collberg: Can you add a short description of these options? Just stating it wouldn't make send to a reviewer. Else, how about having a table for this? Transformation | Options | Description}
\subsubsection{Virtualize}
Virtualize involves transforming a function into an interpretor, whose bytecode language is specialized for the particular function specified. 



\begin{itemize}
\item \textit{--VirtualizeDispatch=[direct,indirect and switch]}.
\item \textit{--VirtualizeSuperOpsRatio=[2 and 0]}.
\item \textit{--VirtualizeMaxMergeLength=[5 and 0]}.
\item \textit{--VirtualizePerformance=[AddressSizeShort, CacheTop and PointerStack]}.
\item \textit{--VirtualizeOperands=[stack and registers]}.
\end{itemize}


\subsubsection{AddOpaque}
Split up control flow by adding opaque branches.

\begin{itemize}
\item \textit{--AddOpaqueKinds=[bug and junk]}
\item \textit{--AddOpaqueCount=[1,5,10,15 and 20]}
\end{itemize}


\subsubsection{Encode Arithmetic}
Encode arithmetic replaces integer arithmetic expression with more complex expression. Tigress currently supports one option for this transformation.

\begin{itemize}
\item \textit{--EncodeArithmetic}.
\end{itemize}


\subsubsection{Encode Literal}
This technique obfuscates string and integer literals with less obvious expression. A single option of encoding integer and string is utilized.

\begin{itemize}
\item \textit{--EncodeLiterals=[Integer and String]}.
\end{itemize}

\subsubsection{Flatten}
Remove control flow from a function.

\begin{itemize}
\item \textit{--FlattenDispatch=[switch, goto and indirect]}.
\item \textit{--FlattenSplitBasicBlocks=[true and false]}.
\end{itemize}



\subsubsection{Merge}
This transformation technique merges multiple functions together into one. The following options are supported in Tigress: 

\begin{itemize}
\item \textit{--MergeFlatten=[true and false]}.
\item \textit{--MergeFlattenDispatch=[goto,indirect and switch]}.
\item Split \textit{--MergeSplitBasicBlocks=[true and false]}.
\item Randomize \textit{--MergeSplitBasicBlocks	=[true and false]}.
\end{itemize}


\subsubsection{Split}
Split transformation breaks a function into less conspicuous pieces that are their own functions. The option utilized for this transformation is an attempt at the number of times the splitting needs to be performed. 

\begin{itemize}
\item \textit{--SplitCount=[1,5,10 and 15]}.
\end{itemize}


\subsection{Benchmarks}
\CC{Jayant: describe the benchmarks here. If we didn't use all the Mibench benchmarks, explain why. If you had to modify the, explain how and why. For example, Tigress requires a merge step, this is important since it can have performance implications (a merged program may optimize better than one with multiple modules.}
\CC{Aakarsh: help out with this since you did the merging etc.}
\CC{Below is the text that Aakarsh wrote, you can use this as a starting point.}

We utilize the widely popular MiBench embedded benchmark suite ~\cite{guthaus2001mibench} as our base programs for obfuscation. Since, MiBench is written in C, it conforms with the requirements of our obfuscator, Tigress, that works only on C code. It is to be noted that since we aim at obfuscating firmware or system code, C is the most common language utilized in the realm of embedded systems. The MiBench provides a range of benchmarks divided into Automative, Network, Office, Security and Telecom applications. We choose to utilize 5 benchmark programs from each of these applications for our obfuscation experiments viz., i)\textit{basicmath} - performs basic mathematical operations for a given set of constants, ii)\textit{FFT} - performs Fast Fourier Transform and it's inverse on an array of data, iii)\textit{SHA} - produces a 160-bit message digest secure hash from a given input, iv)\textit{dijkstra} - constructs a large graph in an adjacency matrix representation and calculates the shortest path between every pair of node, and v)\textit{stringsearch} - searches for given
words in phrases using a case insensitive comparison
algorithm.

The program in the above benchmarks consists of multiple input C files and Tigress requires these input files to be merged into exactly one C file. 

\subsection{Power Measurement and Acquisition Setup}
\begin{figure}[t]
  \centering
\includegraphics[width=0.9\columnwidth, height=8cm]{expsetup}
  \caption{Experimental Framework}\label{expsetup}
\end{figure}

Figure~\ref{expsetup} shows our experimental framework to measure the power and performance characteristics of the obfuscated benchmarks, and the data acquisition system. We target the Raspberry Pi (RPi) which is a popular platform for low-power and low-cost computational tasks~\cite{maksimovic2014raspberry}. Specifically, our measurements are carried out on a Raspberry Pi 3 Model B+ with a Broadcom BCM2837B0 SoC. This is a quad-core A53 (ARMv8) 64-bit 1.4GHz processor with 1GB LPDDR2 SDRAM. It runs on 5V/2.5A DC input power via a microUSB connector. Its operating temperature is between 0 to $60^{\circ}C$.

We have built a custom setup to measure the power consumption from the RPi. We power the RPi via a microUSB from a stable 5V/2.5A power supply. (see Figure~\ref{expsetup}), from which the power consumed by the RPi is $P$ = $V\times I$.
To provide the required power supply to the RPi and capture the power drawn during code execution, we  use the Keysight SMU source meter unit. We use a 2-wire connection with SMU Kelvin probes to an open-ended microUSB 2-wire cable which in turn powers the RPi with constant and accurate power supply. The SMU is connected to a PC via a LAN cable. This is used to control the SMU via Keysight commands to extract the required current and voltage values from the SMU loaded into the Keysight BenchVue Power Supply PC application.

The RPi has an on-chip temperature sensor which measures the temperature of the CPU. This provides additional information of the heat state of the RPi. We used 2 heat-sinks and an external cooling fan in an air conditioned environment to maintain optimal working conditions for the Pi.  

The obfuscated target programs are executed on the RPi and power samples are acquired on the PC. Each experiment is executed 5 times, and execution times and power values are averaged over the runs.
\subsection{Performance acquisition of obfuscated codes}\label{subsec:pdoc}
\CC{Aakarsh and Jayant: Fix this.}
\JP{Fixed.
Note : The Pi 3B+ benefited from a change to the way the system-on-chip (SoC) is attached to the circuit board, allowing it to better dissipate heat. 
And all the factors such as power draw,performance varies with different versions of Pi.
https://www.raspberrypi.org/magpi/raspberry-pi-specs-benchmarks/
}
Target output for each of the obfuscated code is generated by:\\
\textit{gcc -O2 -lm -Wall -o obfuscated\_target\_filename obfuscated\_input \_filename.c}\\

We acquire the execution time(using time command) and temperature of the generated obfuscated code binary (after stripping \footnote{https://linux.die.net/man/1/strip}).

A data file is generated which includes name, start time, average execution time (5 runs), end time and average temperature of RPi3 during execution of obfuscated codes. This data file is fed to Keysight BenchVue Power Supply application which gives output of average power during the code execution. 


\subsection{{Power acquisition of obfuscated codes}}
\JP{Fixed}
We acquire the power values by directly recording current and voltage values from Keysight provided Windows application called as Keysight Benchvue Power Supply. The calculation is done by recording current and voltage values every 0.1 second for the entire 5 times execution of the obfuscated codes.
These values are then averaged and subtracted from the idle RPi3 state power consumption of 2.2805 watts. This gives the exact average power consumed by the particular obfuscated code execution.


\subsection{Energy consumption of obfuscated codes}
\CC{Aakarsh and Jayant: Fix this.}
\JP{Fixed}
The energy consumed is deduced by $P_{avg} \times t_{avg}$, where $P_{avg}$ represents the average power consumed over the execution of the obfuscated target file and $t_{avg}$ is the average execution time. To elaborate, each execution of the obfuscated target file generates several power samples (based on the sampling rate of the SMU(which is 0.1 second in our case), which are averaged to result in $P_{avg}$. 


%\subsection{Evaluation Metrics}
%\CC{Sriram writes this. I'm not sure what's %supposed to be in this section.}



\section{Power Analysis}
\label{sec:analysis}

% Please add the following required packages to your document preamble:
% \usepackage{multirow}

\newcounter{analysisRow}
\setcounter{analysisRow}{1}
\newcommand{\nextRow}{\theanalysisRow\stepcounter{analysisRow}}

\begin{table*}[!hp]
\caption{Energy results. Energy is expressed in Joules.}
\label{tab:results}
\begin{footnotesize}
\begin{tabular}{@{}|p{1.5cm}|l|l|l|p{5cm}|l|p{5cm}|}

%%%%%%%%%%%%%%%%%%%%%%%%%%%%%%%%%%%%%%%%%%%%%%%%%%%%%%%%%%%%%%%%%%%%%%%%%%%%%%%%%%%%%%%%%%%%%%%%%%%%%%%%%%%%%%%%%%%%%%%%
\hline
Benchmark & 
   \#
      & \begin{tabular}[c]{@{}l@{}}
          Obfuscating \\ 
          Transformations
        \end{tabular} 
      & \begin{tabular}[c]{@{}l@{}}
         Min \\ 
         Energy
         \end{tabular} 
      & Min Energy Option 
      & \begin{tabular}[c]{@{}l@{}}
           Max \\ 
           Energy
        \end{tabular} 
      & Max Energy Option \\ \hline
%%%%%%%%%%%%%%%%%%%%%%%%%%%%%%%%%%%%%%%%%%%%%%%%%%%%%%%%%%%%%%%%%%%%%%%%%%%%%%%%%%%%%%%%%%%%%%%%%%%%%%%%%%%%%%%%%%%%%%%%

%%%%%%%%%%%%%%%%%%%%%%%%%%%%%%%%%%%%%%%%%%%%%%%%%%%%%%%%%%%%%%%%%%%%%%%%%%%%%%%%%%%%%%%%%%%%%%%%%%%%%%%%%%%%%%%%%%%%%%%%
\multirow{7}{*}{\begin{tabular}[c]{@{}l@{}}Security - \\ SHA \\ Baseline \\ Unobfuscated \\Energy = 0.06\end{tabular}} 
      %------------------------------------------------------------------------------
      & \nextRow
      & Add Opaque 
      & 0.06 
      & Count=1, Kinds=call, Structs=list 
      & 0.87 
      & Count=20, Kinds=true, Structs=list,array \\ \cline{2-7} 
 
      %------------------------------------------------------------------------------
      & \nextRow
      & EncodeArithmetic 
      & 0.11 
      & Kinds=integer 
      & 0.11 
      & Kinds=integer \\ \cline{2-7} 
 
      %------------------------------------------------------------------------------
      & \nextRow
      & EncodeLiterals 
      & 0.06 
      & Kinds=string 
      & 0.07 
      & Kinds=integer \\ \cline{2-7} 
 
      %------------------------------------------------------------------------------
      & \nextRow
      & Flatten 
      & 0.07 
      & \raggedright Dispatch=goto, SplitBlocks=true, RandomizeBlocks=false, ObfuscateNext=false, 
                    OpaqueStructs=list, ConditionalKinds=branch 
      & 9.38 
      & Dispatch=indirect, SplitBlocks=true, 
        RandomizeBlocks=true, ObfuscateNext=true, OpaqueStructs=array, ConditionalKinds=branch \\ \cline{2-7} 
 
      %------------------------------------------------------------------------------
      & \nextRow
      & Merge 
      & 0.13 
      &  \raggedright Flatten=true, ObfuscateSelect=true, OpaqueStructs=list, Dispatch=goto, RandomizeBlocks=true, 
          SplitBlocks=false, ConditionalKinds=branch 
      & 1.91 
      & Flatten=true, ObfuscateSelect=true, OpaqueStructs=list, 
        Dispatch=indirect, RandomizeBlocks=false, SplitBlocks=true, ConditionalKinds=branch \\ \cline{2-7} 
 
      %------------------------------------------------------------------------------
      & \nextRow
      & Split 
      & 0.04 
      & Count=1, Kinds=block 
      & 0.28 
      & Count=5, Kinds=top \\ \cline{2-7} 
 
      %------------------------------------------------------------------------------
      & \nextRow
      & Virtualize 
      & 13.07 
      & \raggedright Dispatch=switch, SuperOpsRatio=2, MaxMergeLength=5, Performance=PointerStack, Operands=stack 
      & 127.09 
      & Dispatch=switch, SuperOpsRatio=0, MaxMergeLength=0, Performance=PointerStack, Operands=registers \\ \hline
%%%%%%%%%%%%%%%%%%%%%%%%%%%%%%%%%%%%%%%%%%%%%%%%%%%%%%%%%%%%%%%%%%%%%%%%%%%%%%%%%%%%%%%%%%%%%%%%%%%%%%%%%%%%%%%%%%%%%%%%

%%%%%%%%%%%%%%%%%%%%%%%%%%%%%%%%%%%%%%%%%%%%%%%%%%%%%%%%%%%%%%%%%%%%%%%%%%%%%%%%%%%%%%%%%%%%%%%%%%%%%%%%%%%%%%%%%%%%%%%%
\multirow{7}{*}{\begin{tabular}[c]{@{}l@{}}Telecom - \\ FFT \\ Baseline \\ Unobfuscated \\Energy = 0.30\end{tabular}} 
 
      %------------------------------------------------------------------------------
      & \nextRow
      & Add Opaque 
      & 0.31 
      & Count=5, Kinds=bug, Structs=list, array 
      & 1.26 
      & Count=15, Kind=call, Structs=list, array \\ \cline{2-7} 
 
      %------------------------------------------------------------------------------
      & \nextRow
      & EncodeArithmetic 
      & 0.36 
      & Kinds=integer 
      & 0.36 
      & Kinds=integer \\ \cline{2-7} 
 
      %------------------------------------------------------------------------------
      & \nextRow
      & EncodeLiteral 
      & 0.33 
      & Kinds=string 
      & 0.36 
      & Kinds=integer \\ \cline{2-7} 
 
      %------------------------------------------------------------------------------
      & \nextRow
      & Flatten 
      & 0.20 
      & \raggedright Dispatch=switch, SplitBlocks=false, RandomizeBlocks=true, ObfuscateNext=true, OpaqueStructs=array, ConditionalKinds=compute 
      & 1.68 
      & Dispatch=call, SplitBlocks=true, RandomizeBlocks=false, ObfuscateNext=true, OpaqueStructs=array, ConditionalKinds=branch \\ \cline{2-7} 
 
      %------------------------------------------------------------------------------
      & \nextRow
      & Merge 
      & 0.33 
      & \raggedright Flatten=true, ObfuscateSelect=true, OpaqueStructs=list, Dispatch=indirect, RandomizeBlocks=false, SplitBlocks=false, ConditionalKinds=branch 
      & 0.93 
      & Flatten=true, ObfuscateSelect=true, OpaqueStructs=list, Dispatch=switch, RandomizeBlocks=false, SplitBlocks=true, ConditionalKinds=compute \\ \cline{2-7} 
 
      %------------------------------------------------------------------------------
      & \nextRow
      & Split 
      & 0.33 
      & Count=10, SplitKinds=block 
      & 0.72 
      & Count=5, SplitKinds=inside \\ \cline{2-7} 
 
      %------------------------------------------------------------------------------
      & \nextRow
      & Virtualize 
      & 0.83 
      & \raggedright Dispatch=switch, SuperOpsRatio=2, MaxMergeLength=5, Performance=PointerStack, Operands=stack 
      & 14.27 
      &  Dispatch=switch, SuperOpsRatio=2, MaxMergeLength=5, Performance=AddressSizeShort, CacheTop, Operands=registers \\ \hline
%%%%%%%%%%%%%%%%%%%%%%%%%%%%%%%%%%%%%%%%%%%%%%%%%%%%%%%%%%%%%%%%%%%%%%%%%%%%%%%%%%%%%%%%%%%%%%%%%%%%%%%%%%%%%%%%%%%%%%%%

%%%%%%%%%%%%%%%%%%%%%%%%%%%%%%%%%%%%%%%%%%%%%%%%%%%%%%%%%%%%%%%%%%%%%%%%%%%%%%%%%%%%%%%%%%%%%%%%%%%%%%%%%%%%%%%%%%%%%%%%
\multirow{7}{*}{\begin{tabular}[c]{@{}l@{}}Automotive-\\ Qsort \\ Baseline \\ Unobfuscated \\Energy = 0.16 \end{tabular}} 

      %------------------------------------------------------------------------------
      & \nextRow
      & Add Opaque 
      & 0.16 
      & Count=15, Kinds=junk, Structs=list 
      & 0.59 
      & Count=10, Kind=junk, Structs=list \\ \cline{2-7} 
 
      %------------------------------------------------------------------------------
      & \nextRow
      & EncodeArithmetic 
      & 0.23 
      & Kinds=integer 
      & 0.23 
      & Kinds=integer \\ \cline{2-7} 
 
      %------------------------------------------------------------------------------
      & \nextRow
      & EncodeLiteral 
      & 0.22 
      & Kinds=string 
      & 0.26 
      & Kinds=string, integer \\ \cline{2-7} 
 
      %------------------------------------------------------------------------------
      & \nextRow
      & Flatten 
      & 0.11 
      & \raggedright Dispatch=goto, SplitBlocks=true, RandomizeBlocks=false, ObfuscateNext=true, OpaqueStructs=array, ConditionalKinds=compute 
      & 2.01 
      & Dispatch=goto, SplitBlocks=true, RandomizeBlocks=true, ObfuscateNext=true, OpaqueStructs=array, ConditionalKinds=compute \\ \cline{2-7} 
 
      %------------------------------------------------------------------------------
      & \nextRow
      & Merge 
      & 0.16 
      & \raggedright Dispatch=goto, SplitBlocks=true, RandomizeBlocks=true, ObfuscateNext=false, OpaqueStructs=list, ConditionalKinds=branch 
      & 0.47 
      & Dispatch=switch, SplitBlocks=true, RandomizeBlocks=false, ObfuscateNext=false, OpaqueStructs=array, ConditionalKinds=compute \\ \cline{2-7} 
 
      %------------------------------------------------------------------------------
      & \nextRow
      & Split 
      & 0.15 
      & Count=5, SplitKinds=level 
      & 0.26 
      & Count=10, SplitKinds=level 
 \\ \hline
%%%%%%%%%%%%%%%%%%%%%%%%%%%%%%%%%%%%%%%%%%%%%%%%%%%%%%%%%%%%%%%%%%%%%%%%%%%%%%%%%%%%%%%%%%%%%%%%%%%%%%%%%%%%%%%%%%%%%%%%

%%%%%%%%%%%%%%%%%%%%%%%%%%%%%%%%%%%%%%%%%%%%%%%%%%%%%%%%%%%%%%%%%%%%%%%%%%%%%%%%%%%%%%%%%%%%%%%%%%%%%%%%%%%%%%%%%%%%%%%%
\multirow{7}{*}{\begin{tabular}[c]{@{}l@{}}Telecom - \\ CRC32 \\ Baseline \\ Unobfuscated \\Energy = 1.07\end{tabular}} 
 
      %------------------------------------------------------------------------------
      & \nextRow
      & Add Opaque 
      & 1.09 
      & Kind=true, Structs=list 
      & 10.27 
      & Kind=bug, Structs=array \\ \cline{2-7} 
 
      %------------------------------------------------------------------------------
      & \nextRow
      & EncodeArithmetic 
      & 1.57 
      & Kinds=integer 
      & 1.57 
      & Kinds=integer \\ \cline{2-7} 
 
      %------------------------------------------------------------------------------
      & \nextRow
      & EncodeLiteral 
      & 1.46 
      & Kinds=integer 
      & 1.52 
      & Kinds=string, integer \\ \cline{2-7} 
 
      %------------------------------------------------------------------------------
      & \nextRow
      & Flatten 
      & 1.28 
      & \raggedright Dispatch=goto, SplitBlocks=false, RandomizeBlocks=false, ObfuscateNext=false, OpaqueStructs=list, ConditionalKinds=branch 
      & 12.31 
      &  Dispatch=call, SplitBlocks=true, RandomizeBlocks=false, ObfuscateNext=true, OpaqueStructs=array, ConditionalKinds=branch \\ \cline{2-7} 
 
      %------------------------------------------------------------------------------
      & \nextRow
      & Merge 
      & 1.52 
      & \raggedright Flatten=true, ObfuscateSelect=true, OpaqueStructs=list, Dispatch=goto, RandomizeBlocks=false, SplitBlocks=false, ConditionalKinds=branch 
      & 7.44 
      &  Flatten=true, ObfuscateSelect=true, OpaqueStructs=list,array, Dispatch=indirect, RandomizeBlocks=true, SplitBlocks=true, ConditionalKinds=branch \\ \cline{2-7} 
 
      %------------------------------------------------------------------------------
      & \nextRow
      & Split 
      & 1.43 
      & Count=20, SplitKinds=recursive 
      & 3.51 
      & Count=15, SplitKinds=deep \\ \cline{2-7} 
 
      %------------------------------------------------------------------------------
      & \nextRow
      & Virtualize 
      & 7.17 
      & \raggedright Dispatch=switch, SuperOpsRatio=2, MaxMergeLength=5, Performance=PointerStack, Operands=stack 
      & 88.88 
      & Dispatch=switch, SuperOpsRatio=0, MaxMergeLength=0, Performance=AddressSizeShort, CacheTop, Operands=registers \\ \hline
%%%%%%%%%%%%%%%%%%%%%%%%%%%%%%%%%%%%%%%%%%%%%%%%%%%%%%%%%%%%%%%%%%%%%%%%%%%%%%%%%%%%%%%%%%%%%%%%%%%%%%%%%%%%%%%%%%%%%%%%

%%%%%%%%%%%%%%%%%%%%%%%%%%%%%%%%%%%%%%%%%%%%%%%%%%%%%%%%%%%%%%%%%%%%%%%%%%%%%%%%%%%%%%%%%%%%%%%%%%%%%%%%%%%%%%%%%%%%%%%%
\multirow{7}{*}{\begin{tabular}[c]{@{}l@{}}Network - \\ Patricia \\ Baseline \\ Unobfuscated \\Energy = 0.28\end{tabular}} 
 
      %------------------------------------------------------------------------------
      & \nextRow
      & Add Opaque 
      & 0.31 
      & Count=20, Kind=junk, Structs=list 
      & 0.70 
      & Count=20, Kind=true, Structs=array \\ \cline{2-7} 
 
      %------------------------------------------------------------------------------
      & \nextRow
      & EncodeArithmetic 
      & 0.28 
      & Kinds=integer 
      & 0.28 
      & Kinds=integer \\ \cline{2-7} 
 
      %------------------------------------------------------------------------------
      & \nextRow
      & EncodeLiteral 
      & 0.27 
      & Kinds=integer, string 
      & 0.32 
      & Kinds=string \\ \cline{2-7} 
 
      %------------------------------------------------------------------------------
      & \nextRow
      & Flatten 
      & 0.29 
      & \raggedright Dispatch=indirect, SplitBlocks=true, RandomizeBlocks=true, ObfuscateNext=true, OpaqueStructs=array, ConditionalKinds=compute 
      & 0.54 
      & Dispatch=call, SplitBlocks=true, RandomizeBlocks=true, ObfuscateNext=true, OpaqueStructs=array, ConditionalKinds=branch \\ \cline{2-7} 
 
      %------------------------------------------------------------------------------
      & \nextRow
      & Merge 
      & 0.33 
      & \raggedright Flatten=true, ObfuscateSelect=false, OpaqueStructs=list, Dispatch=switch, RandomizeBlocks=false, SplitBlocks=false, ConditionalKinds=branch 
      & 0.51 
      & Flatten=true, ObfuscateSelect=false, OpaqueStructs=array, Dispatch=indirect, RandomizeBlocks=false, SplitBlocks=true, ConditionalKinds=compute \\ \cline{2-7} 
 
      %------------------------------------------------------------------------------
      & \nextRow
      & Split 
      & 0.31 
      & Count=15, SplitKinds=inside 
      & 0.49 
      & Count=10, SplitKinds=deep \\ \cline{2-7} 
%%%%%%%%%%%%%%%%%%%%%%%%%%%%%%%%%%%%%%%%%%%%%%%%%%%%%%%%%%%%%%%%%%%%%%%%%%%%%%%%%%%%%%%%%%%%%%%%%%%%%%%%%%%%%%%%%%%%%%%%
 
\hline
\end{tabular}
\end{footnotesize}
\end{table*}

%\subsection{Performance Analysis}
The immediate impact of the obfuscation transformation seems highest for virtualize in terms of performance due to higher runtimes of code execution. 
It reaches as far as 0.97 times to 19.31 times of increase in average performance to that of unobfuscated code for CRC32. 
It reaches as far as 5.88 times to 91.74 times of increase in average performance to that of unobfuscated code for FTT. 
It reaches as far as 0.62 times to 2.29 times of increase in average performance to that of unobfuscated code for SHA. 
It reaches as far as 0.53 times to 238.66 times of increase in average performance to that of unobfuscated code for Patricia. 
It reaches as far as 0.98 times to 91.74 times of increase in average performance to that of unobfuscated code for Qsort. 


Table~\ref{tab:results} shows the results of our experiments. The 1st column shows the MiBench benchmark, the 2nd column shows the transformation applied. The 3rd column shows the energy used for the transformations options in the 4th column; these are the options to the transformation that gave the {\em lowest} energy increase. Similarly, columns 5 and 6 show the options that had the {\em highest} energy impact.

The results in Table~\ref{tab:results} must be viewed with an understanding that obfuscation tools like Tigress are {\em not} deterministic. To Every invocation 

In terms of highest overall energy impact, virtualize comes first followed by flatten, merge, AddOpaque, Split, EncodeArithmetic and EncodeLiteral.

Virtualize transformation leads to highest impact on energy usage of a application causing as high as 83.26,47.83 and 2239.39 times increase in energy usage of obfuscated code of CRC32,FFT and SHA applications respectively as compared to unobfuscated codes.

The main reason for high energy impact of virtualize is due to high diversity induced in the code after turning the function into an interpreter and highest security against reverse engineering applied to the code. Also other transformations are applied such as split,flatten,merge and addopaque before applying virtualize which increases the complexity. 

Addopaque is also used as a precursor to other transformations. It causes 5 to 10 times increase in energy usage of applications. As more number of opaque predicates are added, the energy usage keeps increasing.

Flatten with dispatch option as call predominantly causes highest energy usage among other variants of flatten. The code sizes of variants with dispatch option as call are also the highest.

Although merge transformation is done after applying flatten to the code, merge incurs lesser upper-bound of energy of usage in most cases. This implies that more number of transformations applied in a sequence does not cause higher energy usage than less or single transformation.

Split transformation upper bounds are generally lower to merge transformation as it leads to increase in count of functions while merge decreases count of functions.

Encode literal and encode arithmatic transformations cause minimal impact on energy which in some cases is lower than unobfuscated code energy usage. This is due to function argument manipulation which is not as complex as other transformations.


As observed from the Table~\ref{tab:results}, certain transformation do not negatively impact the energy usage but instead lead to lesser energy usage in few cases than that of unobfuscated code. For eg., FFT Flatten consumes lesser energy than unobfuscated FFT code. Similarly for SHA Split, Qsort Flatten and Qsort Split, the energy usage is lesser than unobfuscated respective applications.
Depending on the application, the min and max energy usages of transformation variants differ a lot. There is no particular  common pattern common among our benchmark applications which requires deeper analysis into security criticality of the application functions and features.


\CC{All: Here is where we draw our results!! Everybody please contribute!}


%%%%%%%%%%%%%%%%%%%%%%%%%%%%%%%%%%%%
% Old stuff
%%%%%%%%%%%%%%%%%%%%%%%%%%%%%%%%%%%%
\endinput
The immediate impact of the obfuscation transformation seems highest for virtualize in terms of energy usage due to higher execution times and higher power drawn during execution.

It reaches as far as 1.02 to 83.26 times of increase in average energy usage to that of unobfuscated code for CRC32. 
It reaches as far as 0.70 to 47.83 times of increase in average energy usage to that of unobfuscated code for FFT.
It reaches as far as 0.74 to 2239.39 times of increase in average energy usage to that of unobfuscated code for SHA.
It reaches as far as 0.64 to 2.48 times of increase in average energy usage to that of unobfuscated code for Patricia.
It reaches as far as 0.68 to 17.12 times of increase in average energy usage to that of unobfuscated code for Qsort.







\section{Discussion}
\label{sec:discussion}

\CC{Sriram updates this when the paper is nearing completion.}
\AR{This discussion has no association with the context of the paper. Sriram - you may have to update this based on our context/results/analysis}
\emph{\textbf{Application-aware Obfuscation:}}

Our initial analysis revealed that obfuscation incurred a significant increase in energy consumption and execution time. Thus a need for efficient allocation of resources so as to minimize the power consumed due to obfuscation becomes necessary. This is further exacerbated when systems need to meet a given power budget while balancing the performance and power consumption of the devices. Thus, we claim the need for application-aware obfuscation which develops heuristics for obfuscation considering the needs of applications while meeting a power budget. We plan to investigate these ideas as part of future work.  

\emph{\textbf{Energy Optimal Phase Ordering:}}

Obfuscation Executive \cite{heffner} determines the order and number of obfuscating transformations. Further the optimal arrangement of transformations is called the Phase Ordering problem \cite{holder}. Currently, phase ordering problem does not consider the energy consumed by transformations during obfuscation. Thus, to determine the optimal arrangement of transformations for resource-constrained devices, energy consumption needs to be factored. Energy optimal phase ordering can be determined through a brute-force approach which relies on looping through all possible ordering and determining the least energy consuming one. However this approach incurs exponential increase in execution time. Thus, efficient approaches for energy optimal phase ordering are needed.       

\emph{\textbf{Security Approximation:}}

Estimating security of obfuscation is of increasing importance given that the transformations may be vulnerable to multitude of attacks. Thus, there exists a trade-off between protection provided by a transformation and resources expended by adversaries to thwart security. However, analyzing the trade-offs is challenging due to a multitude of transformations coupled with a lack of efficient methods for approximating security of obfuscation techniques. Wu \textit{et al.} \cite{wu2010framework} proposed an approach for security approximation using regression models. However their approach does not consider the numerous transformations and their characteristics present in Tigress. We plan to investiate these ideas as part of future work.  




% Not sure what was supposed tos go here, so commenting it out for now.
%\section{Power Profiling}


\bibliographystyle{ACM-Reference-Format}
\bibliography{code-obfuscation}

\end{document}
