\section{Power Analysis}
\label{sec:analysis}

% Please add the following required packages to your document preamble:
% \usepackage{multirow}
\begin{table*}[!hp]
\caption{Energy results.}
\label{tab:results}
\begin{footnotesize}
\begin{tabular}{|p{1.5cm}|l|l|p{5cm}|l|p{5cm}|}
\hline
Benchmark                                                                      & \begin{tabular}[c]{@{}l@{}}Obfuscation \\ Transformations\end{tabular} & \begin{tabular}[c]{@{}l@{}}Min \\ Energy (J)\end{tabular} & Min Energy Option                                                                                                                                                                          & \begin{tabular}[c]{@{}l@{}}Max \\ Energy (J)\end{tabular} & Max Energy Option                                                                                                                                                                              \\ \hline
\multirow{7}{*}{\begin{tabular}[c]{@{}l@{}}Security - \\ SHA \\ Baseline \\ Unobfuscated \\Energy = 0.0568\end{tabular}}                                                  & Add Opaque                                                             & 0.06                                                    &  Count=1,Kinds=call,Structs=list                                                                                                                                                         & 0.87                                                    &  Count=20,Kinds=true,Structs=list,array                                                                                                                                                      \\ \cline{2-6} 
                                                                               & EncodeArithmetic                                                      & 0.11                                                    &  Kinds=integer                                                                                                                                                                           & 0.11                                                    &  Kinds=integer                                                                                                                                                                               \\ \cline{2-6} 
                                                                               & EncodeLiterals                                                        & 0.06                                                    &  Kinds=string                                                                                                                                                                            & 0.07                                                    &  Kinds=integer                                                                                                                                                                               \\ \cline{2-6} 
                                                                               & Flatten                                                                & 0.07                                                    & \raggedright Dispatch=goto, SplitBasicBlocks=true, RandomizeBasicBlocks=false, ObfuscateNext=false, OpaqueStructs=list, ConditionalKinds=branch              & 9.38                                                    &        Dispatch=indirect, SplitBasicBlocks=true, RandomizeBasicBlocks=true,  ObfuscateNext=true,  OpaqueStructs=array, ConditionalKinds=branch                \\ \cline{2-6} 
                                                                               & Merge                                                                  & 0.13                                                    & 
                                                            Flatten=true, ObfuscateSelect=true, OpaqueStructs=list, Dispatch=goto,  RandomizeBasicBlocks=true, SplitBasicBlocks=false, ConditionalKinds=branch & 1.91                                                    & Flatten=true,ObfuscateSelect=true, OpaqueStructs=list,Dispatch=indirect, RandomizeBasicBlocks=false,S plitBasicBlocks=true, ConditionalKinds=branch \\ \cline{2-6} 
                                                                               & Split                                                                  & 0.04                                                    &  Count=1,Kinds=block                                                                                                                                                                     & 0.28                                                    &  Count=5,Kinds=top                                                                                                                                                                           \\ \cline{2-6} 
                                                                               & Virtualize                                                             & 13.07                                                   & Dispatch=switch, SuperOpsRatio=2, MaxMergeLength=5, Performance=PointerStack, Operands=stack                                                    & 127.09                                                  & Dispatch=switch, SuperOpsRatio=0, MaxMergeLength=0, Performance=PointerStack, Operands=registers                                                    \\ \hline
\multirow{7}{*}{\begin{tabular}[c]{@{}l@{}}Telecom - \\ FFT \\ Baseline \\ Unobfuscated \\Energy = 0.2984\end{tabular}}                                                   & Add Opaque                                                             & 0.31                                                    &  Count=5, Kinds=bug, Structs=list,array                                                                                                                                                  & 1.26                                                    &  Count=15, Kind=call, Structs=list,array                                                                                                                                                     \\ \cline{2-6} 
                                                                               & EncodeArithmetic                                                      & 0.36                                                    &  Kinds=integer                                                                                                                                                                           & 0.36                                                    &  Kinds=integer                                                                                                                                                                               \\ \cline{2-6} 
                                                                               & EncodeLiteral                                                         & 0.33                                                    &  Kinds=string                                                                                                                                                                            & 0.36                                                    &  Kinds=integer                                                                                                                                                                               \\ \cline{2-6} 
                                                                               & Flatten                                                                & 0.20                                                    &  Dispatch=switch, SplitBlocks=false, RandomizeBlocks=true, ObfuscateNext=true, OpaqueStructs=array, ConditionalKinds=compute                                                             & 1.68                                                    &  Dispatch=call, SplitBlocks=true, RandomizeBlocks=false, ObfuscateNext=true, OpaqueStructs=array, ConditionalKinds=branch                                                                    \\ \cline{2-6} 
                                                                               & Merge                                                                  & 0.33                                                    &  Flatten=true, ObfuscateSelect=true, OpaqueStructs=list, Dispatch=indirect, RandomizeBlocks=false, SplitBlocks=false, ConditionalKinds=branch                                            & 0.93                                                    &  Flatten=true, ObfuscateSelect=true, OpaqueStructs=list, Dispatch=switch, RandomizeBlocks=false, SplitBlocks=true, ConditionalKinds=compute                                                  \\ \cline{2-6} 
                                                                               & Split                                                                  & 0.33                                                    &  Count=10, SplitKinds=block                                                                                                                                                              & 0.72                                                    &  Count=5, SplitKinds=inside                                                                                                                                                                  \\ \cline{2-6} 
                                                                               & Virtualize                                                             & 0.83                                                    &  Dispatch=switch, SuperOpsRatio=2, MaxMergeLength=5, Performance=PointerStack, Operands=stack                                                                                            & 14.27                                                   &  Dispatch=switch, SuperOpsRatio=2, MaxMergeLength=5, Performance=AddressSizeShort,CacheTop, Operands=registers                                                                               \\ \hline
\multirow{7}{*}{\begin{tabular}[c]{@{}l@{}}Automotive-\\ Qsort \\ Baseline \\ Unobfuscated \\Energy = 0.16 \end{tabular}}   & Add Opaque                                                             & 0.16                                                    &  Count=15, Kinds=junk, Structs=list                                                                                                                                                      & 0.59                                                    &  Count=10, Kind=junk, Structs=list                                                                                                                                                           \\ \cline{2-6} 
                                                                               & EncodeArithmetic                                                      & 0.23                                                    &  Kinds=integer                                                                                                                                                                           & 0.23                                                    &  Kinds=integer                                                                                                                                                                               \\ \cline{2-6} 
                                                                               & EncodeLiteral                                                         & 0.22                                                    &  Kinds=string                                                                                                                                                                            & 0.26                                                    &  Kinds=string,integer                                                                                                                                                                        \\ \cline{2-6} 
                                                                               & Flatten                                                                & 0.11                                                    &  Dispatch=goto, SplitBlocks=true, RandomizeBlocks=false, ObfuscateNext=true, OpaqueStructs=array, ConditionalKinds=compute                                                               & 2.01                                                    &  Dispatch=goto, SplitBlocks=true, RandomizeBlocks=true, ObfuscateNext=true, OpaqueStructs=array, ConditionalKinds=compute                                                                    \\ \cline{2-6} 
                                                                               & Merge                                                                  & 0.16                                                    &  Dispatch=goto, SplitBlocks=true, RandomizeBlocks=true, ObfuscateNext=false, OpaqueStructs=list, ConditionalKinds=branch                                                                 & 0.47                                                    &  Dispatch=switch, SplitBlocks=true, RandomizeBlocks=false, ObfuscateNext=false, OpaqueStructs=array, ConditionalKinds=compute                                                                \\ \cline{2-6} 
                                                                               & Split                                                                  & 0.15                                                    &  Count=5, SplitKinds=level                                                                                                                                                               & 0.26                                                    &  Count=10, SplitKinds=level                                                                                                                                                                   
                                                                                                                    \\ \hline
\multirow{7}{*}{\begin{tabular}[c]{@{}l@{}}Telecom - \\    CRC32 \\ Baseline \\ Unobfuscated \\Energy = 1.07\end{tabular}} & Add Opaque                                                             & 1.09                                                    &  Kind=true, Structs=list                                                                                                                                                                 & 10.27                                                   &  Kind=bug, Structs=array                                                                                                                                                                     \\ \cline{2-6} 
                                                                               & EncodeArithmetic                                                      & 1.57                                                    &  Kinds=integer                                                                                                                                                                           & 1.57                                                    &  Kinds=integer                                                                                                                                                                               \\ \cline{2-6} 
                                                                               & EncodeLiteral                                                         & 1.46                                                    &  Kinds=integer                                                                                                                                                                           & 1.52                                                    &  Kinds=string,integer                                                                                                                                                                        \\ \cline{2-6} 
                                                                               & Flatten                                                                & 1.28                                                    &  Dispatch=goto, SplitBlocks=false, RandomizeBlocks=false, ObfuscateNext=false, OpaqueStructs=list, ConditionalKinds=branch                                                               & 12.31                                                   &  Dispatch=call, SplitBlocks=true, RandomizeBlocks=false, ObfuscateNext=true, OpaqueStructs=array, ConditionalKinds=branch                                                                    \\ \cline{2-6} 
                                                                               & Merge                                                                  & 1.52                                                    &  Flatten=\_true, ObfuscateSelect=true, OpaqueStructs=list, Dispatch=goto, RandomizeBlocks=false, SplitBlocks=false, ConditionalKinds=branch                                              & 7.44                                                    &  Flatten=true, ObfuscateSelect=true, OpaqueStructs=list,array, Dispatch=indirect, RandomizeBlocks=true, SplitBlocks=true, ConditionalKinds=branch                                            \\ \cline{2-6} 
                                                                               & Split                                                                  & 1.43                                                    &  Count=20, SplitKinds=recursive                                                                                                                                                          & 3.51                                                    &  Count=15, SplitKinds=deep                                                                                                                                                                   \\ \cline{2-6} 
                                                                               & Virtualize                                                             & 7.17                                                    &  Dispatch=switch, SuperOpsRatio=2, MaxMergeLength=5, Performance=PointerStack, Operands=stack                                                                                            & 88.88                                                   &  Dispatch=switch, SuperOpsRatio=0, MaxMergeLength=0, Performance=AddressSizeShort,CacheTop, Operands=registers                                                                               \\ \hline
\multirow{7}{*}{\begin{tabular}[c]{@{}l@{}}Network - \\    Patricia \\ Baseline \\ Unobfuscated \\Energy = 0.28\end{tabular}}                                            & Add Opaque                                                             & 0.31                                                    &  Count=20, Kind=junk, Structs=list                                                                                                                                                       & 0.70                                                    &  Count=20, Kind=true, Structs=array                                                                                                                                                          \\ \cline{2-6} 
                                                                               & EncodeArithmetic                                                      & 0.28                                                    &  Kinds=integer                                                                                                                                                                           & 0.28                                                    &  Kinds=integer                                                                                                                                                                               \\ \cline{2-6} 
                                                                               & EncodeLiteral                                                         & 0.27                                                    &  Kinds=integer,string                                                                                                                                                                    & 0.32                                                    &  Kinds=string                                                                                                                                                                                \\ \cline{2-6} 
                                                                               & Flatten                                                                & 0.29                                                    &  Dispatch=indirect, SplitBlocks=true, RandomizeBlocks=true, ObfuscateNext=true, OpaqueStructs=array, ConditionalKinds=compute                                                            & 0.54                                                    &  Dispatch=call, SplitBlocks=true, RandomizeBlocks=true, ObfuscateNext=true, OpaqueStructs=array, ConditionalKinds=branch                                                                     \\ \cline{2-6} 
                                                                               & Merge                                                                  & 0.33                                                    &  Flatten=true, ObfuscateSelect=false, OpaqueStructs=list, Dispatch=switch, RandomizeBlocks=false, SplitBlocks=false, ConditionalKinds=branch                                             & 0.51                                                    &  Flatten=true, ObfuscateSelect=false, OpaqueStructs=array, Dispatch=indirect, RandomizeBlocks=false, SplitBlocks=true, ConditionalKinds=compute                                              \\ \cline{2-6} 
                                                                               & Split                                                                  & 0.31                                                    &  Count=15, SplitKinds=inside                                                                                                                                                             & 0.49                                                    &  Count=10, SplitKinds=deep                                                                                                                                                                   \\ \cline{2-6} 
                                                                              
\hline
\end{tabular}
\end{footnotesize}
\end{table*}
%\subsection{Performance Analysis}
The immediate impact of the obfuscation transformation seems highest for virtualize in terms of performance due to higher runtimes of code execution. 
It reaches as far as 0.97 times to 19.31 times of increase in average performance to that of unobfuscated code for CRC32. 
It reaches as far as 5.88 times to 91.74 times of increase in average performance to that of unobfuscated code for FTT. 
It reaches as far as 0.62 times to 2.29 times of increase in average performance to that of unobfuscated code for SHA. 
It reaches as far as 0.53 times to 238.66 times of increase in average performance to that of unobfuscated code for Patricia. 
It reaches as far as 0.98 times to 91.74 times of increase in average performance to that of unobfuscated code for Qsort. 


Table~\ref{tab:results} shows the results of our experiments. The 1st column shows the MiBench benchmark, the 2nd column shows the transformation applied. The 3rd column shows the energy used for the options in 4th column; these are the options to the transformation that gave the {\em lowest} energy increase. Similarly, columns 5 and 6 show the options that had the {\em highest} energy impact.

In terms of highest overall energy impact, virtualize comes first followed by flatten, merge, addopaque, split, encodearithmetic and encodeliteral.

Virtualize transformation leads to highest impact on energy usage of a application causing as high as 83.26,47.83 and 2239.39 times increase in energy usage of obfuscated code of CRC32,FFT and SHA applications respectively as compared to unobfuscated codes.

The main reason for high energy impact of virtualize is due to high diversity induced in the code after turning the function into an interpreter and highest security against reverse engineering applied to the code. Also other transformations are applied such as split,flatten,merge and addopaque before applying virtualize which increases the complexity. 

Addopaque is also used as a precursor to other transformations. It causes 5 to 10 times increase in energy usage of applications. As more number of opaque predicates are added, the energy usage keeps increasing.

Flatten with dispatch option as call predominantly causes highest energy usage among other variants of flatten. The code sizes of variants with dispatch as call are also generally the highest.

Although merge transformation is done after applying flatten to the code, merge incurs lesser upper-bound of energy of usage in most cases. This implies that more number of transformations applied in a sequence does not cause higher energy usage than less or single transformation.

Split transformation upper bounds are generally lower to merge transformation as it leads to increase in count of functions while merge decreases count of functions.

Encode literal and encode arithmatic transformations cause minimal impact on energy which in some cases is lower than unobfuscated code energy usage. This is due to function argument manipulation which is not as complex as other transformations.


As observed from the Table~\ref{tab:results}, certain transformation do not negatively impact the energy usage but instead lead to lesser energy usage in few cases than that of unobfuscated code. For eg., FFT Flatten consumes lesser energy than unobfuscated FFT code. Similarly for SHA Split, Qsort Flatten and Qsort Split, the energy usage is lesser than unobfuscated respective applications.
Depending on the application, the min and max energy usages of transformation variants differ a lot. There is no particular pattern in it which requires deeper analysis into security criticality of the application functions  and features of the applications.


\CC{All: Here is where we draw our results!! Everybody please contribute!}


%%%%%%%%%%%%%%%%%%%%%%%%%%%%%%%%%%%%
% Old stuff
%%%%%%%%%%%%%%%%%%%%%%%%%%%%%%%%%%%%
\endinput
The immediate impact of the obfuscation transformation seems highest for virtualize in terms of energy usage due to higher runtimes and higher power drawn during execution.

It reaches as far as 1.02 times to 83.26 times of increase in average energy usage to that of unobfuscated code for CRC32. 
It reaches as far as 0.70 times to 47.83 times of increase in average energy usage to that of unobfuscated code for FFT.
It reaches as far as 0.74 times to 2239.39 times of increase in average energy usage to that of unobfuscated code for SHA.
It reaches as far as 0.64 times to 2.48 times of increase in average energy usage to that of unobfuscated code for Patricia.
It reaches as far as 0.68 times to 17.12 times of increase in average energy usage to that of unobfuscated code for Qsort.





