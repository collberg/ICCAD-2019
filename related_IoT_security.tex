\subsection{Energy impact of Security for IoT Devices}

Ensuring security and privacy on devices that are constrained by power, execution time, and code size is challenging~\cite{7823334, 6970594, 5675772}. The performance impact of security protocols and cryptographic algorithms executing on constrained devices has received much attention~\cite{potlapally2003analyzing, 5983970, 1347774, 5940923}. These works analyze the impact of energy, power and performance in the context of mobile devices where battery life is the main constraint. For example. Potlapally et al.~\cite{potlapally2003analyzing} present a framework for investigating energy requirements of the SSL protocol, and asymmetric, symmetric, and hash algorithms on battery-constrained systems. The goal of their analysis of these algorithms is to develop energy-efficient implementations. The goal of the work presented here is similar, but applied to general purpose code obfuscation, rather than cryptographic algorithms.

%%%%%%%%%%%%%%%%%%%%%%%%
\endinput
This is the foundational idea for our work in analyzing the energy impact of various obfuscation techniques for IoTs.
\AR{Fixed}


%\CC{Aakarsh needs to update this. You need to say things like "In [], Bob investigates ..., and finds that ...}

The same concerns prevail while utilizing obfuscation techniques in such constraint devices, particularly IoTs, that handle sensitive data and run proprietary code requiring the need for code/data obfuscation to harden reverse engineering.