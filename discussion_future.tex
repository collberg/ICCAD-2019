\subsection{Future Work}
We believe it to be essential that a specialized benchmark suite be constructed for software protection research. The reason is that, for real world scenarios, an entire application is seldom obfuscated. Rather, the security-sensitive parts are obfuscated heavily (as heavily as possible given the performance constraints), whereas the remainder of the code is either left un-obfuscated, or receives a thin layer of obfuscation to blur the lines between security-sensitive and regular code. Current benchmarks, such as MiBench, are not written with this in mind. Thus, in the work presented here, we attempted to obfuscate the {\em entire} benchmark application, since MiBench has no notion of what code, if any, is security-sensitive. 

In the future, we would like to construct a benchmark suite that consists of applications that are both security-sensitive and performance-critical. The code of such applications should furthermore be annotated with desired performance/security trade-offs so that obfuscators can select the appropriate transformation for different parts of the code.

A second issue that needs to be addressed is the lack of diversity in hardware platforms considered. In this work we only target the Raspberry Pi and in reference~\cite{raj2017modelling}, Raj et al. 
only target the STM32 Arm Cortex micro-controller. It may well be that certain transformations perform particularly poorly or well for certain IoT platforms, and this needs to be investigated as well.

Software developers who desire to protect their applications through obfuscation typically have little understanding of different types transformations, and certainly cannot be expected to make rational selections of transformations to meet their performance and security goals. The ``holy grail'' of software protection research is to construct an {\em Obfuscation Executive} (OE)~\cite{heffner,Holder:2017}, a tool that will help developers by making such decisions automatically. An OE would take into account the type of asset that is being protected (an algorithm, a cryptographic key, a security check, etc.), the code surrounding the asset, and the desired security and performance goals when determining an optimal sequence of  obfuscating transformations. The results presented here can be used to prime such a tool with energy data but more research is needed to relate that to the level of security achieved. 
