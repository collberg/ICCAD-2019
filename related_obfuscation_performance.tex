\subsection{Energy impact of Code Obfuscation}
\CC{I rewrote this section. The last
paragraph still needs a sentence: ``the authors find...".}
\AR{Fixed}

Little work has been done analyzing the impact of individual obfuscating code transformations on energy use on constrained devices. The work most closely related to the experiments presented here are references~\cite{6976079, dhukovic2015load, raj2017modelling}. 

In~\cite{6976079}, Sahin et al. examine the energy usage of different code obfuscation tools on energy usage for mobile applications. They conclude that code obfuscation has only minor impact on battery life of mobile applications. However, their work is limited to a few  commercial obfuscation tools that mostly perform transformations unlikely to have much impact on performance, such as variable renaming, simple optimizations, and string encryption. Furthermore, they are targeting Java applications, rather than native code applications.

In~\cite{raj2017modelling}, Raj et al. 
examine the impact of code obfuscation on a very low-power micro controller, the STM32 Arm Cortex. They obfuscate four applications from the Mibench suite, using {\em lexical} transformations (identifier renaming), data transformations, and control flow transformations (flattening, deadcode insertion, extended loop condition, loop transdformation). The benchmarks are obfuscated by hand. The authors find that when the transformations are combined, the energy consumptions increases by up to 14\%.

In~\cite{dhukovic2015load}, Dukovi\'{c} and Varga present a methodology for evaluating the power overhead of code obfuscation using {\em load profiles}.
They use three commercial obfuscators that operate on the assembly code level, which perform transformations such as {\em instruction substitution}, {\em dead code insertion}, and {\em code transposition}. The authors analyze the power consumption pattern of code obfuscation using two benchmarks and conclude {\em Smart Assembly} to be the best code obfuscator to utilize if power is the criteria with {\em dead code insertion} being the main contribution to the power.   

\endinput

rather than a taxonomy of obfuscation techniques, and ii) they do not analyze the detailed impact of a suite of available code obfuscation techniques ~\cite{tigress} to suggest particular obfuscation techniques that can be utilized for resource-constrained devices. 
In this paper, we fill this gap by analyzing the energy impact of various obfuscation techniques in order to suggest the use of specific obfuscation techniques provided a given energy constraint.  
