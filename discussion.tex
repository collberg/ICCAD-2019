
\section{Discussion}
\label{sec:discussion}

\CC{Sriram updates this when the paper is nearing completion.}
\AR{This discussion has no association with the context of the paper. Sriram - you may have to update this based on our context/results/analysis}
\emph{\textbf{Application-aware Obfuscation:}}

Our initial analysis revealed that obfuscation incurred a significant increase in energy consumption and execution time. Thus a need for efficient allocation of resources so as to minimize the power consumed due to obfuscation becomes necessary. This is further exacerbated when systems need to meet a given power budget while balancing the performance and power consumption of the devices. Thus, we claim the need for application-aware obfuscation which develops heuristics for obfuscation considering the needs of applications while meeting a power budget. We plan to investigate these ideas as part of future work.  

\emph{\textbf{Energy Optimal Phase Ordering:}}

Obfuscation Executive \cite{heffner} determines the order and number of obfuscating transformations. Further the optimal arrangement of transformations is called the Phase Ordering problem \cite{holder}. Currently, phase ordering problem does not consider the energy consumed by transformations during obfuscation. Thus, to determine the optimal arrangement of transformations for resource-constrained devices, energy consumption needs to be factored. Energy optimal phase ordering can be determined through a brute-force approach which relies on looping through all possible ordering and determining the least energy consuming one. However this approach incurs exponential increase in execution time. Thus, efficient approaches for energy optimal phase ordering are needed.       

\emph{\textbf{Security Approximation:}}

Estimating security of obfuscation is of increasing importance given that the transformations may be vulnerable to multitude of attacks. Thus, there exists a trade-off between protection provided by a transformation and resources expended by adversaries to thwart security. However, analyzing the trade-offs is challenging due to a multitude of transformations coupled with a lack of efficient methods for approximating security of obfuscation techniques. Wu \textit{et al.} \cite{wu2010framework} proposed an approach for security approximation using regression models. However their approach does not consider the numerous transformations and their characteristics present in Tigress. We plan to investiate these ideas as part of future work.  


