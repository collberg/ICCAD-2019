Previous work on evaluating the energy impact from code obfuscation has made serious simplifying assumptions: Sahin et al.~\cite{6976079} consider a few commercial obfuscation tools that only provide trivial transformations, Raj et al.~\cite{raj2017modelling} obfuscate by hand which prevents them from examining a large corpus of benchmarks and applying complex transformations, and Dukovi\'{c} and Varga~\cite{dhukovic2015load} target three commercial obfuscators that also only support trivial transformations such as {\em instruction substitution}. Unfortunately, the real world is much messier than this: industry standard software protection tools (such as those sold by IRDETO and Arxan\footnote{\url{https://irdeto.com} and \url{https://www.arxan.com}.}) support large numbers of transformations, each with multiple options, and with the ability to apply multiple layers of transformations to prevent a variety of reverse engineering attacks. 
The design of Tigress is similar: it supports over 30 transformations and over 200 options to these transformations.

This means that any type of evaluation---performance or security---becomes difficult: for a ``real'' obfuscation tool there is no single ``obfuscation'' to evaluate, but rather an infinite space of possible sequences of transformations, each of which can be individually tweaked. Furthermore, the impact of a particular transformation on a particular application depends not only on the transformation, but on the structure of the application: applying the Flatten transformation to a straight-line program will have no impact, and neither will applying EncodeArithmetic to a program that has no integer arithmetic operations.

The problem is further exacerbated by the fact that the {\em real} question we want to have answered is not ``which sequence of obfuscating transformations will have the smallest performance impact,'' but rather ``which sequence of transformations will result in the smallest runtime, code size, and energy use, while maximizing resistance to reverse engineering attacks.'' {\em This}, of course, requires us to be able to measure the resilience to attack of individual and combined transformations which, in itself, is a problem that has yet to receive a satisfactory solution~\cite{banescu2017characterizing,wu2010framework}.

Never-the-less, in this paper we have made a first step towards a methodology for obfuscation performance evaluation. We have targeted a ``real'' obfuscation tool (Tigress) which, although nowhere near as extensive as those provided commercially, shares its architecture and transformation-types with commercial tools. We have applied the transformations supplied by Tigress to applications from the standard MiBench benchmark suite, exhaustively varying the most important transformation options. And we have showed that, while transformations like Flatten and Virtualize can have serious impact on energy usage, by carefully tweaking the options to these transformations, energy overhead can be minimized.