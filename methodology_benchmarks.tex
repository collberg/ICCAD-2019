\subsection{Benchmarks}
\CC{Jayant: describe the benchmarks here. If we didn't use all the Mibench benchmarks, explain why. If you had to modify the, explain how and why. For example, Tigress requires a merge step, this is important since it can have performance implications (a merged program may optimize better than one with multiple modules.}
\CC{Aakarsh: help out with this since you did the merging etc.}
\CC{Below is the text that Aakarsh wrote, you can use this as a starting point.}
\JP{Made changes}

We utilize the widely popular MiBench embedded benchmark suite ~\cite{guthaus2001mibench} as our base programs for obfuscation. Since, MiBench is written in C, it conforms with the requirements of our obfuscator, Tigress, that works only on C code. It is to be noted that since we aim at obfuscating firmware or system code, C is the most common language utilized in the realm of embedded systems. The MiBench provides a range of benchmarks divided into Automative, Network, Office, Security and Telecom applications. We choose to utilize 5 benchmark programs from these applications for our obfuscation experiments viz., i)\textit{CRC32} -performs a 32-bit Cyclic Redundancy Check on a file, ii)\textit{FFT} - performs Fast Fourier Transform and it's inverse on an array of data, iii)\textit{SHA} - produces a 160-bit message digest secure hash from a given input. iv)\textit{qsort} -qsort test sorts a large array of strings into ascending order using the well known quick sort algorithm., and v)\textit{Patricia}- patricia tries are used to represent routing tables in network applications. The input data for this benchmark is a list of IP traffic from a highly active web server for a 2 hour period where The IP numbers are disguised.
We were able to apply all 7 transformations to all functions of the applications except \textit{Virtualize} for \textit{patricia} and \textit{qsort} where single function only could be obfuscated as Tigress is in experimental stage for ARMV8 processor architecture of our IoT device. 

The program in the above benchmarks consists of multiple input C files and Tigress requires these input files to be merged into exactly one C file. 
This is done through merging using CIL merger.
Merging is performed by utilizing \textit{cilly}, the CIL driver \footnote{https://people.eecs.berkeley.edu/~necula/cil/merger.html}. The generalized command for merging is as follows://
\textit{cilly --merge [flags] input\_file\_list [-o target\_output\_filename]
--mergedout=merged\_c \_filename.c}
