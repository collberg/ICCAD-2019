\subsection{Benchmarks}
\CC{Jayant: describe the benchmarks here. If we didn't use all the Mibench benchmarks, explain why. If you had to modify the, explain how and why. For example, Tigress requires a merge step, this is important since it can have performance implications (a merged program may optimize better than one with multiple modules.}
\CC{Aakarsh: help out with this since you did the merging etc.}
\CC{Below is the text that Aakarsh wrote, you can use this as a starting point.}

We utilize the widely popular MiBench embedded benchmark suite ~\cite{guthaus2001mibench} as our base programs for obfuscation. Since, MiBench is written in C, it conforms with the requirements of our obfuscator, Tigress, that works only on C code. It is to be noted that since we aim at obfuscating firmware or system code, C is the most common language utilized in the realm of embedded systems. The MiBench provides a range of benchmarks divided into Automative, Network, Office, Security and Telecom applications. We choose to utilize 5 benchmark programs from each of these applications for our obfuscation experiments viz., i)\textit{basicmath} - performs basic mathematical operations for a given set of constants, ii)\textit{FFT} - performs Fast Fourier Transform and it's inverse on an array of data, iii)\textit{SHA} - produces a 160-bit message digest secure hash from a given input, iv)\textit{dijkstra} - constructs a large graph in an adjacency matrix representation and calculates the shortest path between every pair of node, and v)\textit{stringsearch} - searches for given
words in phrases using a case insensitive comparison
algorithm.

The program in the above benchmarks consists of multiple input C files and Tigress requires these input files to be merged into exactly one C file. 
