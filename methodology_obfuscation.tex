\subsection{Obfuscating Transformations}



% Please add the following required packages to your document preamble:
% \usepackage{multirow}
\begin{table*}[p]
\caption{Tigress transformations and transformation options used in the experiments.}
\label{tab:tigress}
\begin{tabular}{|p{2.25cm}|p{4cm}|p{11cm}|}
\hline
\multicolumn{1}{|c|}{Transformation} & \multicolumn{1}{c|}{Description}                                                                                                                                                                                                                                                                        & \multicolumn{1}{c|}{Options}                                                                                                                                                                                                                                                                                                                             \\ \hline
\multirow{3}{*}{AddOpaque}           & \multirow{3}{4cm}{Split statement sequences by adding if-statement, protected by opaque predicates~\cite{collberg98manufacturing}.} & \underline{Count} \{1,5,10,15 and 20\}: Number of opaques to add to each function                                                                                                                                                                                                                                                                                             \\ \cline{3-3} 
                                     &                                                                                                                                                                                                                                                                                                         & \underline{Kinds} \{true,bug,call,junk\}: Insertion of bogus computation allowed. Where true is the real statement, bug is a buggy statement, call is a random function and junk is random bytes.                                                                                                                                                                     \\ \cline{3-3} 
                                     &                                                                                                                                                                                                                                                                                                         & \underline{Structs} \{list, array\}: Generate opaque expressions using linked lists or arrays.                                                                                                                                                                                                                                                                       \\ \hline
EncodeArithmetic                     & Replace integer arithmetic expression with more complex expression.                                                                                                                                                                                                                  &                                                                                                                                                                                                                                                                                            \\ \hline
EncodeLiteral                        & Obfuscate string and integer literals with less obvious expressions.                                                                                                                                                                                                                    &                                                                                                                                                                                                                                                                                               \\ \hline
\multirow{6}{*}{Flatten}             & \multirow{6}{4cm}{Remove structured control flow using the control flow flattening~\cite{wang00security}.}                                                                                                                                                                                         & \underline{Dispatch} \{call, goto, switch, and indirect\}: Specifies the dispatch method of the flattened blocks of code. For call, each block is turned into its own function, goto uses indirect goto statements, and switch uses switch statements.                                                                                                       \\ \cline{3-3} 
                                     &                                                                                                                                                                                                                                                                                                         & \underline{SplitBasicBlocks} \{true,false\}: Flatten into smaller pieces by splitting up basic blocks prior to flattening.                                                                                                                                                              \\ \cline{3-3} 
                                     &                                                                                                                                                                                                                                                                                                         & \underline{RandomizeBlocks} \{true,false\}: If true, randomize  basic block sequences.                                                                                                                                                                                                                                                                 \\ \cline{3-3} 
                                     &                                                                                                                                                                                                                                                                                                         & \underline{ObfuscateNext} \{true,false\}: If true, obfuscate the computation of the dispatch variable using opaque expressions.                                                                                                                                                                                                                                        \\ \cline{3-3} 
                                     &                                                                                                                                                                                                                                                                                                         & \underline{OpaqueStructs} \{list,array\}: Type of opaque predicate to use which.                                                                                                                                                                                                                                                       \\ \cline{3-3} 
                                     &                                                                                                                                                                                                                                                                                                         & \underline{ConditionalKinds} \{branch,compute\}: How to transform conditional branches: either normal branches computed jumps.                                                                                                                                                                                                          \\ \hline
\multirow{7}{*}{Merge}               & \multirow{7}{4cm}{Merge multiple functions together into one, adding an extra formal argument to be able to call any of the constituent functions. Useful as a precursor to virtualization.
}                                                                                                                                                                                                             & Flatten \{true\}: Flatten before merging.                                                                                                                                                                                                                                                                                                                \\ \cline{3-3} 
                                     &                                                                                                                                                                                                                                                                                                         & \underline{ObfuscateSelect} \{true,false\}: If true, obfuscate the function selection argument with opaque expressions.                                                                                                                                                                                                               \\ \cline{3-3} 
                                     &                                                                                                                                                                                                                                                                                                         & \underline{OpaqueStructs} \{list,array\}: Type of opaque predicate to use.                                                                                                                                                                                                                                                       \\ \cline{3-3} 
                                     &                                                                                                                                                                                                                                                                                                         & \underline{Dispatch} \{indirect ,goto,switch\}: The dispatch method of the flattened merge.                                                                                                                       \\ \cline{3-3} 
                                     &                                                                                                                                                                                                                                                                                                         & \underline{RandomizeBlocks} \{true,false\}: If true, flattened basic block sequences are randomized.                                                                                                                                                                                                                                                                 \\ \cline{3-3} 
                                     &                                                                                                                                                                                                                                                                                                         & \underline{Split basic blocks} \{true,false\}: If true, basic blocks will be split up into individual blocks prior to merging.                                                                                                                                                                                                                                 \\ \cline{3-3} 
                                     &                                                                                                                                                                                                                                                                                                         & \underline{ConditionalKinds} \{branch,compute\}: How to transform conditional branches: either normal branches computed jumps.                                                                                                                                                                                                          \\ \hline
\multirow{2}{*}{Split}               & \multirow{2}{4cm}{Break a large function into smaller, less conspicuous, pieces which can be further obfuscated. }                                                                                                                                                                                                           & \underline{Count} \{1,10,15 and 20\}: Number of attempts to split the function.                                                                                                                                                                                                                                                                   \\ \cline{3-3} 
                                     &                                                                                                                                                                                                                                                                                                         & \underline{SplitKinds} \{block,inside,deep,top and recursive\}: \\ \hline
\multirow{5}{*}{Virtualize}          & \multirow{5}{4cm}{Turn a function into an interpreter. The bytecode language is unique to this function. 
}                                                                                                                                                                                                                                                  & \underline{Dispatch} \{switch and indirect\}: The interpreter's dispatch method. {\em Indirect} uses {\tt gcc}'s {\em labels-as-values} to implement {\em indirect threaded dispatch}.                                                                                                                                                                                                                                                                               \\ \cline{3-3} 
                                     &                                                                                                                                                                                                                                                                                                         & \underline{SuperOpsRatio} \{0,2\}: Number of super operators~\cite{proebsting96optimizing}. If {\tt >0} randomly merges  instructions together to make highly diverse virtual instruction sets.                                                                                                                                                                                                                                                                                                      \\ \cline{3-3} 
                                     &                                                                                                                                                                                                                                                                                                         & \underline{MaxMergeLength} \{0,5\}: Longest sequence of instructions to be merged into one with superoperators.                                                                                                                                                                                                                                                                        \\ \cline{3-3} 
                                     &                                                                                                                                                                                                                                                                                                         & \underline{Performance} \{PointerStack,AddressSizeShort,CacheTop\}: Tweak performance:  AddressSizeShort uses pointer arithmetic to store instruction handler addresses in a 16-bit variable, CacheTop stores the top of stack in a register.                                                                  \\ \cline{3-3} 
                                     &                                                                                                                                                                                                                                                                                                         & \underline{Operands} \{stack, registers\}: Types of virtual instruction set operands: either (implicit) stack operands or (explicit) register operands, or both. Increases virtual instruction set diversity.                                                                                                                                                                                               \\ \hline
\end{tabular}
\end{table*}

In our experiments we use the Tigress C source-to-source obfuscator. Tigress supports a wide range of transformations, and each transformation has multiple options that controls how the obfuscated code is generated. Table~\ref{tab:tigress} shows the Tigress transformations, options, and range of option values that we used in our experiments.

\begin{figure*}
\begin{center}
\begin{minipage}{12cm}
\begin{lstlisting}[basicstyle=\footnotesize]
tigress --Seed=42 --Transform=InitOpaque --Functions=main \
   --Transform=Virtualize --Functions=foo --VirtualizeDispatch=direct \
   --Transform=AddOpaque --Functions=foo --AddOpaqueCount=20 --out=output.c input.c
\end{lstlisting}
\end{minipage}
\end{center}
\caption{Example Tigress invocation.}
\label{fig:tigresscall}
\end{figure*}

Figure~\ref{fig:tigresscall} shows a  call to Tigress to transform a C program {\tt input.c} into {\tt output.c}. This particular invocation first applies the {\tt Virtualize} transformation to the function named {\tt foo} using {\em direct} dispatch, and then adds 20 opaque predicates to the virtualized {\tt foo}. This is a very typical use of an obfuscator, applying multiple transformations to the same function, in order to protect against different types of attack. Note that a {\em seed} can be given to Tigress, which randomizes all internal decisions and allows multiple, differently obfuscated, programs to be generated from the same input program. This is used to generate a diverse set of programs, making it more difficult for an adversary to create generalized attack scripts that work across many obfuscated programs. 

\CC{AR and JP: are these numbers still correct?}
\JP{Dear sir, these are the exact numbers matching with our variants but not the same as before in aakarshs paper}

In our experiments, we do not combine transformations, but only apply one at a time to each of the benchmarks. For each transformation we run all possible  combinations of the values of all options shown in  Table~\ref{tab:tigress}. This results in 16 variant options for \emph{Virtualize}, 128 variant options for \emph{Merge}, 35 variant options for \emph{Split}, 75 variant options for \emph{Add Opaque}, 128 variant options for \emph{Flatten}, 1 variant option for \emph{Encode Arithmetic} and 3 variant options for \emph{Encode Literal}. 

We apply each transformation variant to all the functions in each benchmark. In practice, this is unrealistic; in most scenarios only security-sensitive functions are obfuscated to avoid too much overhead. For the MiBench benchmarks, however, it is unclear which part of the code should be considered ``security-sensitive,'' and we therefore transform the entire program.

\CC{AR and JP: fix this with the correct command and compiler, including compiler version.}
After the obfuscated programs have been generated they are compiled with optimization using the {\tt gcc} compiiler: {\tt gcc -O i.c ....}.

%%%%%%%%%%%%%%%%%%%%%%%%%%%%%%%%%%%%%%
% Old stuff
%%%%%%%%%%%%%%%%%%%%%%%%%%%%%%%%%%%%%%
\endinput

\AR{Dr.Collberg: Can you add a short description of these options? Just stating it wouldn't make send to a reviewer. Else, how about having a table for this? Transformation | Options | Description}
\subsubsection{Virtualize}
Virtualize involves transforming a function into an interpretor, whose bytecode language is specialized for the particular function specified. 



\begin{itemize}
\item \textit{--VirtualizeDispatch=[direct,indirect and switch]}.
\item \textit{--VirtualizeSuperOpsRatio=[2 and 0]}.
\item \textit{--VirtualizeMaxMergeLength=[5 and 0]}.
\item \textit{--VirtualizePerformance=[AddressSizeShort, CacheTop and PointerStack]}.
\item \textit{--VirtualizeOperands=[stack and registers]}.
\end{itemize}


\subsubsection{AddOpaque}
Split up control flow by adding opaque branches.

\begin{itemize}
\item \textit{--AddOpaqueKinds=[bug and junk]}
\item \textit{--AddOpaqueCount=[1,5,10,15 and 20]}
\end{itemize}


\subsubsection{Encode Arithmetic}
Encode arithmetic replaces integer arithmetic expression with more complex expression. Tigress currently supports one option for this transformation.

\begin{itemize}
\item \textit{--EncodeArithmetic}.
\end{itemize}


\subsubsection{Encode Literal}
This technique obfuscates string and integer literals with less obvious expression. A single option of encoding integer and string is utilized.

\begin{itemize}
\item \textit{--EncodeLiterals=[Integer and String]}.
\end{itemize}

\subsubsection{Flatten}
Remove control flow from a function.

\begin{itemize}
\item \textit{--FlattenDispatch=[switch, goto and indirect]}.
\item \textit{--FlattenSplitBasicBlocks=[true and false]}.
\end{itemize}



\subsubsection{Merge}
This transformation technique merges multiple functions together into one. The following options are supported in Tigress: 

\begin{itemize}
\item \textit{--MergeFlatten=[true and false]}.
\item \textit{--MergeFlattenDispatch=[goto,indirect and switch]}.
\item Split \textit{--MergeSplitBasicBlocks=[true and false]}.
\item Randomize \textit{--MergeSplitBasicBlocks	=[true and false]}.
\end{itemize}


\subsubsection{Split}
Split transformation breaks a function into less conspicuous pieces that are their own functions. The option utilized for this transformation is an attempt at the number of times the splitting needs to be performed. 

\begin{itemize}
\item \textit{--SplitCount=[1,5,10 and 15]}.
\end{itemize}

