\subsection{Performance acquisition of obfuscated codes}\label{subsec:pdoc}
\CC{Aakarsh and Jayant: Fix this.}
\JP{Fixed.
Note : The Pi 3B+ benefited from a change to the way the system-on-chip (SoC) is attached to the circuit board, allowing it to better dissipate heat. 
And all the factors such as power draw,performance varies with different versions of Pi.
https://www.raspberrypi.org/magpi/raspberry-pi-specs-benchmarks/
}
Target output for each of the obfuscated code is generated by:\\
\textit{gcc -O2 -lm -Wall -o obfuscated\_target\_filename obfuscated\_input \_filename.c}\\

We acquire the execution time(using time command) and temperature of the generated obfuscated code binary (after stripping \footnote{https://linux.die.net/man/1/strip}).

A data file is generated which includes name, start time, average execution time (5 runs), end time and average temperature of RPi3 during execution of obfuscated codes. This data file is fed to Keysight BenchVue Power Supply application which gives output of average power during the code execution. 


\subsection{{Power acquisition of obfuscated codes}}
\JP{Fixed}
We acquire the power values by directly recording current and voltage values from Keysight provided Windows application called as Keysight Benchvue Power Supply. The calculation is done by recording current and voltage values every 0.1 second for the entire 5 times execution of the obfuscated codes.
These values are then averaged and subtracted from the idle RPi3 state power consumption of 2.2805 watts. This gives the exact average power consumed by the particular obfuscated code execution.


\subsection{Energy consumption of obfuscated codes}
\CC{Aakarsh and Jayant: Fix this.}
\JP{Fixed}
The energy consumed is deduced by $P_{avg} \times t_{avg}$, where $P_{avg}$ represents the average power consumed over the execution of the obfuscated target file and $t_{avg}$ is the average execution time. To elaborate, each execution of the obfuscated target file generates several power samples (based on the sampling rate of the SMU(which is 0.1 second in our case), which are averaged to result in $P_{avg}$. 
