\subsection{Power acquisition of obfuscated codes}\label{subsec:pdoc}
\CC{Aakarsh and Jayant: Fix this.}

Target output for each of the obfuscated code is generated by:\\
\textit{gcc -O2 -lm -Wall -o obfuscated\_target\_filename obfuscated\_input\_filename.c}\\
We acquire the number of lines of code of the generate obfuscated code and size of the generated obfuscated target outputs (after stripping \footnote{https://linux.die.net/man/1/strip}). The generated obfuscated target files are run in order to acquire the run time and temperature of their execution. The Raspberry Pi 2 has an on-chip temperature sensor that measures the temperature (T) of the CPU. This provides additional information of energy consumption ($E$) characteristics of obfuscation techniques as $\Delta T \propto E$. The temperature and run time values are averaged over 10 executions of the obfuscated targets.

\subsection{Energy consumption of obfuscated codes}
\CC{Aakarsh and Jayant: Fix this.}
The energy consumed is deduced by $P_{avg} \times t_{avg}$, where $P_{avg}$ represents the average power consumed over the execution of the obfuscated target file and $t_{avg}$ is the average execution time. To elaborate, each execution of the obfuscated target file generates several power samples (based on the sampling rate of the Arduino), which are averaged to result in $P_{avg}$. An instance of this calculation is illustrated in Fig.~\ref{enercal}
\begin{figure}[t]
  \hspace{-0.2in}
%  \includegraphics[width=3.1in, height=2in]{energy_cal.png}
  \caption{Energy Calculation Instance}\label{enercal}
\end{figure}

