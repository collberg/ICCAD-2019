\begin{abstract}
Code obfuscation provides time-limited protection of the confidentiality of data and software by transforming code into code that is difficult to analyze, thus making reverse engineering harder. While the literature is rife with obfuscating code transformations, their application within the context of embedded systems poses considerable challenges, due to critical resource constraints, especially energy. In this paper we study the energy and performance impact of a wide range of obfuscation techniques for embedded devices. Specifically, we use Tigress, a freely available source-to-source C language obfuscator and characterize the power consumed by transformations applied to the Mibench benchmark. Our study reveals that, while energy consumption can increase as much as 2000 times over unobfuscated code for the popular virtualization transformation, careful selection of obfuscation parameters can reduce this overhead significantly. The insights gained enable software developers to select appropriate obfuscating transformations depending on their energy budget. 
\end{abstract}