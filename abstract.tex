\begin{abstract}
Code obfuscation provides time-limited protection of the confidentiality of data and software by garbling code, thus making reverse engineering harder. While the literature is rife with obfuscating code transformations, their application within the context of embedded systems poses considerable challenges, due to critical resource constraints, especially energy. In this paper we present a comprehensive study of the energy and performance impact of a wide range of obfuscation techniques for embedded devices. In particular, we leverage Tigress, a freely available source-to-source C language obfuscator and characterize the power consumed by transformations applied to the Mibench benchmark suite and comparatively analyze energy-performance trade-offs. Our initial studies reveal that, energy consumption and execution time increase by a maximum of twenty-fold and eight-fold respectively for the virtualization transformation compared to unobfuscated code. These insights enable software developers and software vendors to select appropriate obfuscating transformations depending on the energy, storage, and performance requirements of embedded applications. 
\end{abstract}