\section{Introduction}
\CC{Christian is in the process of rewriting the introduction.}
% MATE attacks against IoT devices
The Internet of Things (IoT) has seen an unprecedented growth, representing a new paradigm of devices interacting with each other, their environment, and the larger Internet~\cite{ATZORI20102787}. As these embedded devices have become ubiquitous, security and privacy have emerged as critical concerns~\cite{weber2010internet,7054433}. Since IoT devices run propriety software and firmware and handle sensitive data, they become susceptible to attacks through tampering and reverse engineering. These types of attacks, where an adversary is in physical control of an IoT device and can manipulate its code, data, or hardware at will, are popularly termed {\em Man-At-The-End} (MATE) attacks. Techniques to mitigate such attacks are termed {\em software protection} or {\em anti-tamper protection}~\cite{falcarin2011guest}.

% Software protection techniques
Obfuscation and tamperproofing are popular techniques to protect code and data against MATE attacks~\cite{collberg_surreptitious_2010}. Experimental assessment of obfuscation to protect against MATE attacks has shown positive outcomes in~\cite{5090041,7781792}. It has been argued~\cite{Hosseinzadeh2015} that obfuscation can enhance code and data protection in the context of resource constrained devices as well. However, application of obfuscation in such environments pose challenges, given the power, performance, and code size constraints. Some prior work has addressed these issues (see references \cite{6976079,dhukovic2015load,raj2017modelling}), but only in the context of simplistic code obfuscation techniques.

Given the limited computational power available on IoT devices, any technique that purports to achieve secure computation on such a device must be evaluated with respect to its overhead (power, performance, and code size). Several works exist in literature that have analyzed and addressed the impact of security mechanisms (cryptographic algorithms and security protocols) in such devices.    

. Both open-source and commercial tools have been used to study the overall affect of obfuscation on energy, efficiency and quality of code. A taxonomy of well established obfuscation techniques was postulated~\cite{collberg1997taxonomy}. Little is studied of the impacts of these obfuscation techniques on resources relative to each other.\\
\indent In our work, we study the relative impact of various obfuscation transformation techniques on power and performance in a resource constraint environment like IoTs. This enables the selection of the right obfuscation technique and parameters provided the constraints of the system and the input software/data for protection. The Raspberry Pi (RPi) is a fitting representative of such an environment as it consumes very less power (idle state of $\sim$1.2W), connected to the Internet and includes input-output pins for interacting with the environment. Tigress\footnote{http://tigress.cs.arizona.edu/}~\cite{Collberg2012Distributed} is a diversified C obfuscator that has been built for the RPi ARM architecture that supports a wide range of obfuscation transformations and options. In our experimental setup we utilize the RPi as our IoT and Tigress as our obfuscation tool for the MiBench embedded benchmark suite~\cite{990739}. Tigress is apt for this environment because MiBench is written in C and also much of embedded and IoT software and/or firmware is developed in C. We build a custom power measurement and data acquisition framework to measure the current drawn and in-turn power consumed by the various obfuscation techniques.

The rest of the paper is organized as follows. Section~\ref{sec:rw} discusses the related work. Section~\ref{sec:po} describes the preliminaries of obfuscation and the techniques utilized for this work. Section~\ref{sec:ef} details the experimental framework setup for power measurement and acquisition. We discuss our results in Section~\ref{sec:res} and summarize the conclusions in Section~\ref{sec:con}.

