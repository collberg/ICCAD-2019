\section{Introduction}
% MATE attacks against IoT devices
The Internet of Things (IoT) has seen an unprecedented growth, representing a new paradigm of devices interacting with each other, their environment, and the larger Internet~\cite{ATZORI20102787}. As these embedded devices have become ubiquitous, security and privacy have emerged as critical concerns~\cite{weber2010internet,7054433}. Since IoT devices run proprietary software and firmware and handle sensitive data, they become susceptible to attacks through tampering and reverse engineering. These types of attacks, where an adversary is in physical control of an IoT device and can manipulate its code, data, or hardware at will, are termed {\em Man-At-The-End} (MATE) attacks. 

% Software protection techniques.
Techniques to mitigate MATE attacks are termed {\em software protection} or {\em anti-tamper protection}~\cite{falcarin2011guest}. {\em Code Obfuscation} and {\em Software Tamperproofing} are popular techniques to protect code and data against MATE attacks~\cite{collberg_surreptitious_2010}. Experimental assessment of obfuscation has shown that it provides (time-limited~\cite{hohl98time}) protection against MATE attacks~\cite{5090041,7781792}.

% Performance issues for IoT devices.
It has been argued~\cite{Hosseinzadeh2015} that obfuscation can enhance code and data protection in the context of resource constrained devices as well. However, given the limited computational power available on IoT devices, any technique that purports to achieve secure computation on such devices must be evaluated with respect to its overhead: power, performance, and code size. This is a serious concern for obfuscation-based protection, as many obfuscating transformations can result in considerable computational overhead. Some prior work has addressed these issues in the context of simplistic code obfuscation techniques~\cite{6976079,dhukovic2015load,raj2017modelling}.
\subsection{}
% What we do in this work.
In this work, we study the relative impact of different obfuscating transformations on power and performance in a resource constrained environment. The ultimate goal is to make available data that will allow practitioners to select the right obfuscation technique given (a) the constraints of their hardware; (b) the characteristics of their software system; and (c) the type of asset they want to protect. Specifically, in this paper we present measurements of the energy consumption of the Raspberry Pi (RPi) running programs from the MiBench benchmark suite that have been obfuscated with the Tigress code obfuscation tool. 

Tigress~\cite{Collberg2012Distributed,banescu2015framework,banescu2016code} is a diversifying C-to-C source code obfuscator that supports a wide range of traditional obfuscating transformations. In our experimental setup we utilize the RPi as our IoT device and Tigress as our obfuscation tool. We use the MiBench embedded benchmark suite~\cite{990739} which is written in C and is designed to be representative of embedded software. We build a custom power measurement and data acquisition framework to measure the current drawn and power consumed by the various obfuscation techniques.

% Overview of rest of the paper
The rest of the paper is organized as follows. In Section~\ref{sec:related} we discuss related work in code obfuscation and IoT security. Section~\ref{sec:methodology} we describe our experimental methodology, including the obfuscating transformations we employ, and details of the experimental framework setup for power measurement and acquisition. We discuss our results in Section~\ref{sec:analysis} and outline directions for future work in Section~\ref{sec:discussion}.

% Sharing statement
{\em Sharing Statement:} The binary for the \tigress obfuscation tool can be downloaded from \tigressurl (source code is available to researchers on request). All source code, scripts, and benchmarks used to derive the results in this paper, along with raw and processed data sets are freely available to the research community at \sourceurl. 

%%%%%%%%%%%%%%%%%%%%%%%%%%%%%%%%%%%%%%%
% Old stuff
%%%%%%%%%%%%%%%%%%%%%%%%%%%%%%%%%%%%%%%
\endinput
Several works exist in literature that have analyzed and addressed the impact of security mechanisms (cryptographic algorithms and security protocols) in such devices. \CC{CITATION}   

Both open-source and commercial tools have been used to study the overall affect of obfuscation on energy, efficiency and quality of code. A taxonomy of well established obfuscation techniques was postulated~\cite{collberg1997taxonomy}. Little is studied of the impacts of these obfuscation techniques on resources relative to each other.


